\documentclass[10pt]{article}
\usepackage{lmodern}
\usepackage[T1]{fontenc}

\usepackage{fullpage}

\usepackage{graphicx}
\graphicspath{ {./images/} }

\usepackage{color}
\usepackage{soul}

\title{\Huge{MIS}}
\author{by \LARGE{\emph{Swapnil}}\\ \ \\ written in {\LaTeX}}

\begin{document}
\maketitle
\large

\section{Types of Information System}
There are three types of information system.
\begin{enumerate}
  \item \textbf{Transaction Processing System}

    The term "transaction processing system" refers to an information system that 
    processes data as originating from business transactions. The primary purpose 
    of transaction processing system is to offer transactions to update records 
    and produce reports required for store keeping.

  \item \textbf{Management Information System}

    The purpose of a management information system is to transform comparatively 
    raw data accessible through using Transaction Processing System into a summarized
    and aggregated form for managers, generally is the form of a report.

  \item \textbf{Decision Support System}

    DSS is interactive, which offers information, data manipulation tools and models
    to support decision-making in a semi-structured and unstructured scenario.

  \item \textbf{Expert System}

    The expert system contains expertise which is helpful for a manager in identifying
    problems or in problem-solving. The principle of artificial intelligence research 
    are used to develop these kinds of information systems.

\end{enumerate}


\section{How MIS helps to grow business? OR How to solve business problem with 
information system? OR Explain the benefits of Information System in business.}
Even the simplest use of technology can dramatically improve the business productivity
and efficiency. The greatest benefit of information systems is their ability to give 
users the information they need to carry out tasks efficiently.

Information system can produce:
\begin{itemize}
  \item custom data to help with a specific task or decision making.

  \item custom formats (\textit{lists, charts, etc.}) which can be tailored to the user's 
    needs.

  \item real-time data, particularly useful where fast action is needed (example: \textit
    {mechanical fault})

  \item archived data, particularly useful for report analysis and business planning.

\end{itemize}

To maximise the benefits of your IT system, we have to fully utilize all its features
and functions. Here we can use:
\begin{itemize}
  \item use instant messaging, emails, voice and video calls and even chatbot technology
    to improve communication with customers and suppliers. This could save our time, money
    and effort, allowing us to reach quickly to new work.

  \item integrate various IT system to reduce administrative costs.

  \item use labelling products with unique numbers and scannable barcodes to boost your 
    efficiency and improve stock control and supply chain management.

  \item use different solutions, such as customer management systems or mobile technologies
    to improve levels of customer service. These may help us to record organize and plan 
    contact with customers, access customer details on the go and view customer interactions.
  
\end{itemize}

\section{Advantages of MIS}
\begin{enumerate}
  \item \textbf{Helps in Managing Data}

    MIS helps in maintaining and managing crucial business data for assisting in complex 
    decision-making by the management.
  
  \item \textbf{Analyzes Trends}

    Management needs to prepare forecasts for strategic planning and determine future goals.
    Thus, to create such a strategy, it is essentials to have accurate reports on prevailing
    market trends. MIS uses various mathematical tools for analyzing the current market trends 
    and predicting future trends based on such information. 

  \item \textbf{Helps in strategic planning}

    MIS reports play a significant role in the strategic planning of the company. It helps 
    in determining the future needs of the company and assists in formulating goals and 
    strategy based on such info.

  \item \textbf{Problems Identification}

    MIS report provides information related to every aspect of activities taking place in the 
    company.

  \item \textbf{Increases Efficiency}

    The relevant information provided by MIS and reporting is utilized in formulating the goals
    and strategy of the company,

  \item \textbf{Comparison of business performance}
    
    The relevant business data and information of the company is stored and maintained in the 
    MIS database.

\end{enumerate}


\section{Write short notes on Decision Support System (DSS).}
Decision support systems (DSS) are interactive software-based systems intended to help manager 
in decision-making by accessing large volumes of information generated from various related
information systems involved in organizational business processes, such as office automation
systems, transaction processing system, etc.

Characteristics of DSS are:
\begin{itemize}
  \item Adaptability and flexibility
  \item High level of interactivity
  \item Ease of use
  \item Efficiency and effectiveness
  \item Complete control by decision-makers
  \item Ease of development
  
\end{itemize}
\section{Difference between efficiency \& effectiveness.}
\resizebox{\columnwidth}{!}{%
\begin{tabular}{cll}
                       & \multicolumn{1}{c}{\textbf{Efficiency}}                                                                                                                                                                                                             & \multicolumn{1}{c}{\textbf{Effectiveness}}                                                                                                                                   \\ \cline{2-3} 
\multicolumn{1}{c|}{1} & \multicolumn{1}{l|}{\begin{tabular}[c]{@{}l@{}}Efficiency refers to the act of performing\\ activities with minimum wastage of time\\ and optimum usage of resources so that\\ the work done is faster and in an error\\ free manner.\end{tabular}} & \multicolumn{1}{l|}{\begin{tabular}[c]{@{}l@{}}Effectiveness is the extent to which\\ someone or something is successful\\ towards meeting the desired outcome\end{tabular}} \\ \cline{2-3} 
\multicolumn{1}{c|}{2} & \multicolumn{1}{l|}{Doing the assigned task in a correct way.}                                                                                                                                                                                      & \multicolumn{1}{l|}{Doing the assigned task accurately.}                                                                                                                     \\ \cline{2-3} 
\multicolumn{1}{c|}{3} & \multicolumn{1}{l|}{\begin{tabular}[c]{@{}l@{}}Efficiency is focused on the input and\\ output.\end{tabular}}                                                                                                                                       & \multicolumn{1}{l|}{\begin{tabular}[c]{@{}l@{}}Effectiveness is focused on the extent\\ to which work is done and the end\\ result achieved\end{tabular}}                    \\ \cline{2-3} 
\multicolumn{1}{c|}{4} & \multicolumn{1}{l|}{Efficiency is effort oriented}                                                                                                                                                                                                  & \multicolumn{1}{l|}{Effectiveness is not effort oriented}                                                                                                                    \\ \cline{2-3} 
\multicolumn{1}{c|}{5} & \multicolumn{1}{l|}{Efficiency is more operation oriented}                                                                                                                                                                                          & \multicolumn{1}{l|}{Effectiveness is more strategy oriented}                                                                                                                 \\ \cline{2-3} 
\multicolumn{1}{c|}{6} & \multicolumn{1}{l|}{Efficiency is time oriented}                                                                                                                                                                                                    & \multicolumn{1}{l|}{Effectiveness is not time oriented.}                                                                                                                     \\ \cline{2-3} 
\end{tabular}%
}



\end{document}
