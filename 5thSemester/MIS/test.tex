\documentclass{memoir}
\usepackage{hyperref}
\usepackage{longtable}

\begin{document}
  
All the importants of Management information system from Rajib Sir at
one place.

\hypertarget{unit-1}{%
\subsection{Unit 1}\label{unit-1}}

\hypertarget{what-is-information-system}{%
\subsubsection{What is information
system?}\label{what-is-information-system}}

An information system is a formal, sociotechnical, organizational system
designed to collect, process, store, and distribute information. From a
sociotechnical perspective, information systems are composed by four
components: task, people, structure, and technology.

\hypertarget{types-of-information-system}{%
\paragraph{{[}{[}Types of information
System{]}{]}:}\label{types-of-information-system}}

\begin{enumerate}
\item Transaction Processing System
\item Management Information System
\item Decision Support System
\item Expert System
\end{enumerate}

\hypertarget{write-a-short-note-on-decision-support-system.}{%
\subsubsection{Write a short note on Decision support
system.}\label{write-a-short-note-on-decision-support-system.}}

A decision support system (DSS) is an information system that supports
business or organizational decision-making activities. DSSs serve the
management, operations and planning levels of an organization (usually
mid and higher management) and help people make decisions about problems
that may be rapidly changing and not easily specified in advance -- IE
unstructured and semi-structured decision problems. Decision support
systems can be either fully computerized or human-powered, or a
combination of both.

Some important characteristics of DSS are: - Adaptability and
flexibolity - High level of interactivity - Ease of use - Efficiency and
effectiveness - Complete control by decision-makers - Ease of
development

\hypertarget{how-information-system-helps-to-grow-business}{%
\subsubsection{How Information System helps to grow
business?}\label{how-information-system-helps-to-grow-business}}

With constant change and evolution of customer preferences and
requirements, businesses need to bring out new methods and innovative
techniques to survive the market and continue to function as per the
customer demands. Implementation of an Information System in such cases
could help in controlling the internal and external processes.

Benefits of Information System are as follows:

\hypertarget{new-products-and-services}{%
\paragraph{New Products and Services}\label{new-products-and-services}}

Businesses need a well organized business information system. It helps
in analyzing independent processes and enables organized work
activities. Also allows companies to understand how to develop, sell
services or products.

\hypertarget{information-storage}{%
\paragraph{Information Storage}\label{information-storage}}

This helps in keep logs of different `important' activities so, in case
of a problem, people can make better decisions. Business information
system makes it easier to store operational data. revision histories,
different records and documents. Generally, the information is stored in
a database which makes data retrieving and processing easier.

\hypertarget{simplified-decision-making}{%
\paragraph{Simplified Decision
Making}\label{simplified-decision-making}}

It simplifies the process of decision making and simplifies the process
of delivering the required information which helps us to take better
decisions.

\hypertarget{behavioral-change}{%
\paragraph{Behavioral Change}\label{behavioral-change}}

It is effective in implementing better communication between the
employers and the employees. Information systems store documents and
files in a common database which can be accessed and shared by the
employees. It helps oversee the flow of information between the
management and the lower-level employees. Allows the front-line
employees to be a part of the decision making process and hence feel
motivated and committed towards doing a task.

\hypertarget{difference-between-tps-and-mis}{%
\subsection{\#\#\# Difference between TPS and
MIS}\label{difference-between-tps-and-mis}}

\begin{longtable}[]{@{}
  >{\raggedright\arraybackslash}p{(\columnwidth - 4\tabcolsep) * \real{0.3333}}
  >{\raggedright\arraybackslash}p{(\columnwidth - 4\tabcolsep) * \real{0.3333}}
  >{\raggedright\arraybackslash}p{(\columnwidth - 4\tabcolsep) * \real{0.3333}}@{}}
\toprule
\begin{minipage}[b]{\linewidth}\raggedright
\end{minipage} & \begin{minipage}[b]{\linewidth}\raggedright
TPS
\end{minipage} & \begin{minipage}[b]{\linewidth}\raggedright
MIS
\end{minipage} \\
\midrule
\endhead
Input & Transaction/events & Output from TPS \\
Output & Data entry, listing, sorting, merging and updating. & Routine
reports, simple models, low level analysis. \\
Users & Operational personals, lowerlevel managers, supervisors &
Middle-level manager. \\
Goal & Records and processes transactions. & Production of summary and
exception reports. \\
Decision and support & Provides decision support to lower-level
managers. & Provides decision supports to tactical-level managers. \\
\bottomrule
\end{longtable}
\end{document}
