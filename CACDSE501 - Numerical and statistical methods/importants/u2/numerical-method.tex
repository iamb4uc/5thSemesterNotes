\documentclass{article}

% {{{ Packages
        \usepackage[margin=1in]{geometry}                           % Margins
          \linespread{1.2}                                          % Increases line spacing
        \usepackage{graphicx}                                       % Image Support
          \graphicspath{ {./images/} }                              % Path to image directory
        \usepackage{color}                                          % Colors for stuff
        \usepackage{soul}                                           % Provides Highlight feature
        \usepackage{fancyhdr}                                       % Fancy Header and footer
        \usepackage{amsmath}                                        % Math support
        \usepackage{amssymb}                                        % Extended symbols for math
        \usepackage{etoolbox}                                       % IDK what it does. I just coppied it from stackexchage
          \AtBeginEnvironment{align}{\setcounter{equation}{0}}      % resets align counter for each instance
        \usepackage{listings}                                       % Code-snippet support
          % \definecolor{dkgreen}{rgb}{0,0.6,0}                     % lst-configs
          % \definecolor{gray}{rgb}{0.5,0.5,0.5}
          % \definecolor{mauve}{rgb}{0.58,0,0.82}

          % \lstset{                                                % lst-format-config
          %   frame=tb,
          %   language=Java,
          %   aboveskip=3mm,
          %   belowskip=3mm,
          %   showstringspaces=false,
          %   columns=flexible,
          %   basicstyle={\small\ttfamily},
          %   numbers=none,
          %   numberstyle=\tiny\color{gray},
          %   keywordstyle=\color{blue},
          %   commentstyle=\color{dkgreen},
          %   stringstyle=\color{mauve},
          %   breaklines=true,
          %   breakatwhitespace=true,
          %   tabsize=3
          % }
        \usepackage{caption}                                      % Caption support
        \usepackage[utf8]{inputenc}                               % utf encoding
        \usepackage{hyperref}
          \hypersetup{pdfnewwindow=true, colorlinks=false}
% }}}

% {{{ Footer section
        % Creates footer
        \pagestyle{fancy}%
        \fancyhf{}%
        \lfoot{Swapnil}%
        \cfoot{iamb4uc.xyz}%
        \rfoot{Page \thepage}%
        \renewcommand{\headrulewidth}{0pt}% Line at the head invisible
        \renewcommand{\footrulewidth}{0.9pt}% Line at the footer visible
% }}}

% {{{ Title
        \title{\Huge{\texttt{Numerical and Statistical Method}}}
        \author{\huge{by \emph{Swapnil}}}
        \date{}
% }}}

\begin{document}
%{{{ Title page
      \maketitle
      \thispagestyle{empty}
      % \newpage
% }}}

%{{{ ToC
% }}}



%%%%%%%%%%%%%%%%%%%%%%%%%%%%%%%%%%%%%%%%%%%%%%%%%%%%%%%%%%%%%%%%%%%%%%%%%%%%%%%%%%%%%%%%%%%
%%%%%%%%%%%%%%%%%%%%%%%%%%%%%%%%%%%%%% Document Body %%%%%%%%%%%%%%%%%%%%%%%%%%%%%%%%%%%%%%
%%%%%%%%%%%%%%%%%%%%%%%%%%%%%%%%%%%%%%%%%%%%%%%%%%%%%%%%%%%%%%%%%%%%%%%%%%%%%%%%%%%%%%%%%%%
      % {{{
    \section{Lagrenge's interpolation formula}
    Let, $f(x_0),f(x_1),\dots,f(x_n)$ be the entries corresponding to the argument 
    $x_0,x_1,\dots,x_n$ which are not necessarily in equal intervals. The interpolation 
    formula given by Lagrenge to estimate the value of $y$ corresponding to a value of $x$ 
    between any two consecutive values of the given values is

    \begin{align*}
      f(x)&=\frac{(x-x_1)(x-x_2)\dots(x-x_n)}{(x_0-x_1)(x_0-x_2)\dots(x_0-x_n)}f(x_0)+\frac{(x-x_0)(x-x_2)\dots(x-x_n)}{(x_1-x_0)(x_1-x_2)\dots(x_1-x_n)}f(x_1)\\
      &+ \dots +\frac{(x-x_1)(x-x_2)\dots(x-x_{n-1})}{(x_n-x_0)(x_n-x_1)\dots(x_n-x_{n-1})}f(x_n)
    \end{align*}

    Lagrange's Interpolation formula is usually and when the values of the independent
    variable $x$ are not equidistant. To apply Lagrange's formula the value of $x$ 
    corresponding to which the value of $y$ is to be interpolated may be anywhere in 
    between the first and last terms.
    
    To apply Lagrange's formula, constuction of difference table is not needed.
  
  \subsection{Trapezoidal Rule}
    The general quadrature formula is

    \begin{align*}
      I &=\int_{a=x_0}^{b=x_0+nh}f(x)dx\\
        &= h[ny_0+\frac{n^2}{2}\Delta y_0+(\frac{n^3}{3}-\frac{n^2}{2})\frac{\Delta^2y_0}{2!}+(\frac{n^4}{4}-n^3+n^2)\frac{\Delta^3y_0}{3!}
    \end{align*}

    \begin{align}\label{dv3}
        &+ \dots+upto\ (n+1)\ terms]
    \end{align}
    
    Putting $n=1$ in equation number \ref{dv3}, and neglecting all the differece 
    higher than 1, we get:
    
    \begin{align*}
      I_1 &= \int_{x_0}^{x_0+h}f(x)dx\\
          &= h[y_0+\frac{1}{2}\Delta y_0]\\
          &= h[y_0+\frac{1}{2}(y_1-y_0)]\\
          &= h[\frac{2y_0+y_1-y_0}{2}]\\
          &= h[\frac{y_0+y_1}{2}]
    \end{align*}
    
    Similarly,
    
    \begin{align*}
      I_2 &= \int_{x_0+h}^{x_0+2h}f(x)dx = h[\frac{y_1+y_2}{2}]
    \end{align*}
    
    \begin{align*}
      I_3 &= \int_{x_0+2h}^{x_0+3h}f(x)dx = h[\frac{y_2+y_3}{2}]
    \end{align*}
    \dots 
    
    \dots 
    \begin{align*}
      I_n &= \int_{x_0+(n-1)h}^{x_0+nh}f(x)dx = h[\frac{y_{(n-1)}+y_n}{2}]
    \end{align*}
    
    Adding these n integrals we get
    \begin{align*}
      \int_{x_0}^{x_0+nh}f(x)dx &= h[(\frac{y_0-y_1}{2})+(\frac{y_1-y_2}{2})+\dots+(\frac{y_{(n-1)}+y_n}{2})]\\
                                &= h[(\frac{y_0+y_n}{2})+(y_1+y_2+y_3+\dots+y_{n-1})]
    \end{align*}
    
    This is called the Trapezoidal Rule.
    
    \subsection{Condition for validity of trapezoidal rule}
    \begin{enumerate}
      \item $f(x)$ should be a polynomial of degree of 1, i.e. a 
        straight line of the form $f(x)=a+bx$
      \item The value of $x$ should be equidistant
      \item The number of divisions of the integral $(x_0,x_0+nh)$ 
        should be multiple of 1, like 3, 5, 7, etc.
    \end{enumerate}
  
  \section{Simpson's $\frac{1}{3}^{rd}$ rule}
    The general quadrature formula is 

    \begin{align*}
      I &=\int_{a=x_0}^{b=x_0+nh}f(x)dx\\
        &= h[ny_0+\frac{n^2}{2}\Delta y_0+(\frac{n^3}{3}-\frac{n^2}{2})\frac{\Delta^2y_0}{2!}+(\frac{n^4}{4}-n^3+n^2)\frac{\Delta^3y_0}{3!}\\
        &+ \dots+upto\ (n+1)\ terms]
    \end{align*}

    Putting $n=2$ and neglecting all the differences higher then second order.

    \begin{align*}
      I_1=\int_{a=x_0}^{b=x_0+2h}f(x)dx &= h[2y_0+2\Delta y_0+(\frac{8}{3}-2)\frac{\Delta^2y_0}{2!}]\\
                                        &= h[2y_0+2(y_1-y_0)+(\frac{8}{3}-2)\frac{y_2-2y_1+y_0}{2}]\\
                                        &= h[2y_0+2y_1-2y_0+\frac{2}{3}*\frac{y_2-2y_1+y_0}{2}]\\
                                        &= h[2y_1+\frac{y_2-2y_1+y_0}{3}]\\
                                        &= h[\frac{6y_1+y_2-2y_1+y_0}{3}]\\
                                        &= h[\frac{y_0+4y_1+y2}{3}]\\
                                        &= \frac{h}{3}(y_0+4y_1+y_2)
    \end{align*}
    
    Simillarly,
    
    \begin{align*}
      I_4=\int_{x_0+2h}^{x_0+4h}f(x)dx &= \frac{h}{3}(y_2+4y_3+y_4)
    \end{align*}

    And,

    \begin{align*}
      I_n=\int_{x_0+(n-2)h}^{x_0+nh}f(x)dx &= \frac{h}{3}(y_{n-2}+4y_{n-1}+y_n)
    \end{align*}
    
    On adding all these integrals, we get:
    
    \begin{align*}
      \int_{x_0}^{x_0+nh}f(x)dx &= \frac{h}{3}(y_0+y_n)+4(y_1+y_3+\dots+y_{n-1})+2(y_2+y_4+\dots+y_{n-2})
    \end{align*}

    This formula is known as Simpson's $\frac{1}{3}^{rd}$ formula.
    
    \subsection{Condition for validity of Simpson's $\frac{1}{3}^{rd}$ rule.}

    \begin{enumerate}
      \item $f(x)$ should be a polynomial of degree 2 of the form $f(x)=a+bx+cx^2$
      \item The values of x should be equidistant
      \item The number of division of the interval $(x_0,x_0+nh)$ should be multiple of 2 like 2, 4, 6, 8, 10, etc.
    \end{enumerate}
  
  \section{Simpson's $\frac{3}{8}^{th}$ rule}
    The general quadrature formula is 

    \begin{align*}
      I &=\int_{a=x_0}^{b=x_0+nh}f(x)dx\\
        &= h[ny_0+\frac{n^2}{2}\Delta y_0+(\frac{n^3}{3}-\frac{n^2}{2})\frac{\Delta^2y_0}{2!}+(\frac{n^4}{4}-n^3+n^2)\frac{\Delta^3y_0}{3!}\\
        &+ \dots+upto\ (n+1)\ terms]
    \end{align*}

    Putting $n=3$ and neglicting all the difference higher than $3^{rd}$ order.

    \begin{align*}
      I_1=\int_{x_0}^{x_0+3h}f(x)dx &= h[3y_0+\frac{3}{2}\Delta y_0+(\frac{27}{3}-\frac{9}{2})\frac{\Delta^2y_0}{2!}+(\frac{81}{4}-27+9)\frac{\Delta^3y_0}{3!}]\\
                              &= h[3y_0+\frac{3}{2}(y_1-y_0)+(9-\frac{9}{2})(\frac{y_2-2y_1+y_0}{2})+(\frac{81}{4}-27+9)\\
                              &  (\frac{y_3-3y_2+3y_1-y_0}{6})]\\
                              \ \\ 
                              &= h[3y_0+\frac{9}{2}(y_1-y_0)+\frac{9}{3}(\frac{y_2-2y_1+y_0}{2})+\frac{3}{8}(y_3-3_2+3y_1-y_0)]\\
                              &= \frac{3h}{8}[8y_0+12(y_1-y_0)+6(y_2-2y_1+y_0)+(y_3-3y_2+3y_1-y_0)]\\
                              &= \frac{3h}{8}[y_0+3y_1+3y_2+y_3]\\
    \end{align*}
    
    Similarly,
    
    \begin{align*}
      \int_{x_0+3h}^{x_0+6h}f(x)dx &= \frac{3h}{8}[y_3+3y_4+3y_5+y_6]\\
    \end{align*}
    \dots
    
    \dots
    \begin{align*}
      \int_{x_0+(n-3)h}^{x_0+nh}f(x)dx &= \frac{3h}{8}[y_{n-3}+3y_{n-2}+3y_{n-1}+y_n]\\
    \end{align*}
    
    Adding all the integrals,
    
    \begin{align*}
      \int_{x_0}^{x_0+nh}f(x)dx=\frac{3h}{8}[(y_0+y_n) &+ 3(y_1+y_2+y_4+y_5+\dots+y_{n-1})\\
                                                       &+ 2(y_3+y_6+\dots+y_{n-3})]
    \end{align*}

    This formula is known as Simpson's $\frac{3}{8}^{rd}$ formula.
    
    \subsection{Conditions for validity of Simpson's $\frac{3}{8}^{rd}$ rule.}

    \begin{enumerate}
      \item $f(x)$ should be a polynomial of degree 3 of the form 
        $f(x)=a+6x+cx^2+dx^3$
      \item The value of $x$ should be equidistant.
      \item The number of division of the interval $(x_0,x_0+nh)$
        should be multiple of 3 like 3, 6, 9, etc.
    \end{enumerate}

    \section{Bi-section method}
    This method of solving a trascendental equation consists in locating the roots of the equation $f(x)=0$ between two number, say $x_0$ and $x_1$, such that $f(x)$ is continuous for $x_0\leq x\leq x_1$ and $f(x_0)$ and $f(x_1)$ are the opposite signs so that $f(x_0)*f(x_1)<0$ IE, the curve
    crosses the x-axis between $x_0$ and $x_1$ and the desire root of the given equation is approximately,
    \begin{align*}
               x_2 &= \frac{x_0+x_1}{2}\\
               x_3 &= \frac{x_0+x_2}{2},\text{provided $f(x_0)*f(x_2)<0$}\\
                   &= \frac{x_1+x_2}{2},\text{provided $f(x_1)*f(x_2)<0$}\\
      Further\ x_4 &= \frac{x_0+x_3}{2},\text{provided $f(x_0)*f(x_3)<0$}\\
      \text{and so on\dots}
    \end{align*}
    Thus on each iteration, we either find the root with desired accuracy or we narrow down the range to half of the previous interval.

    % }}}


%%%%%%%%%%%%%%%%%%%%%%%%%%%%%%%%%%%%%%%%%%%%%%%%%%%%%%%%%%%%%%%%%%%%%%%%%%%%%%%%%%%%%%%%%%%%
\end{document}
