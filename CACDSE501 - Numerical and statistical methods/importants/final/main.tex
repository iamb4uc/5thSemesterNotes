\documentclass[11pt, a4paper]{article}

%{{{
% Packages{{{
\usepackage{mathpazo}
\usepackage{todonotes}
\usepackage{multirow}
\usepackage{graphicx}
  \graphicspath{ {./images/} }
\usepackage[T1]{fontenc}
  \linespread{1.4}
\usepackage{microtype}
\usepackage{amsmath}
\usepackage{amssymb}
\usepackage{etoolbox}
\AtBeginEnvironment{align}{\setcounter{equation}{0}}
\usepackage[english]{babel}

\usepackage[margin=0.7in]{geometry}

\usepackage{enumitem}
  \setlist[enumerate]{itemsep=0px, left=0pt}
  \setlist[itemize]{itemsep=0px, left=0pt}

\usepackage{titlesec}
\titleformat{\part}[block]{\Huge\slshape\bfseries\centering}{}{0em}{}
\titleformat{\section}[block]{\LARGE\scshape\bfseries\centering}{}{0em}{}
\titleformat{\subsection}[block]{\Large\bfseries}{}{0em}{}
\titleformat{\subsubsection}[block]{\bfseries}{}{0em}{}
\usepackage{titling}
\usepackage{tcolorbox}

\usepackage{fancyhdr} % Headers and footers
\pagestyle{fancy} % All pages have headers and footers
\fancyhead{} % Blank out the default header
\fancyfoot{} % Blank out the default footer
\renewcommand{\headrule}{}
%}}}

% Title{{{
\setlength{\droptitle}{-4\baselineskip}

\pretitle{\begin{center}\Huge\bfseries}
\posttitle{\end{center}}
\title{Numerical Methods}
\author{ by
  \textsc{Swapnil}%\\[1px]
}
\date{}
%}}}}}}

\begin{document}
\maketitle
\tableofcontents
\thispagestyle{empty}
\newpage
\parindent0pt

\section{Calculas of Finite difference}%{{{
The calculas of finite difference deals with the changes in the values of the 
changes in the values of the dependent variable due to the changes in the 
independent variable.

Let $y$ be a function of $x$ presented by $y=f(x)$.

Let $a, a+h, a+2h, \dots, (a+nh)$ be the $n+1$ equidistant values of $x$ and
$f(a), f(a+h), f(x+2h), \dots, f(a+nh)$ be the corresponding values of $f(x)$.
The values of $x$ are called arguments and the values of $f(x)$ are called
entries.

Again,
$$
(a+h)-a=(a+2h)-(a+h)\dots(a+nh)-\{ a+(n-1) \} = h
$$
where h is called the interval of difference.
\subsection{Difference of first order}
$$
\Delta f(x)=f(x+h)-f(x)
$$
\subsection{Difference of second order}
\begin{align*}
  \Delta^2f(x) &= \Delta[\Delta f(x)]\\
               &= \Delta[f(x+h)-f(x)]\\
               &= f(x+h)-\Delta f(x)\\
               &= f(x+2h)-f(x+h)-[f(x+h)-f(x)]\\
\end{align*}

\begin{tcolorbox}
  \begin{center}

    $\Delta^2f(x) = f(x+2h)-2f(x+h)+f(x)-h$

  \end{center}
\end{tcolorbox}

\subsection{Difference of third order}
\begin{align*}
  \Delta^3f(x) &= \Delta[\Delta^2f(x)]\\
               &= \Delta[f(x-2h)-2f(x+h)+f(x)]\\
               &= \Delta f(x+2h)-2\Delta f(x+h) + \Delta f(x)\\
               &= f(x+3h)-f(x+2h)-2f(x+2h)+2f(x+h)+f(x+h)-f(x)\\
\end{align*}

\begin{tcolorbox}
  \begin{center}

    $\Delta^3f(x) = f(x+3h)-3f(x+2h)+3f(x+h)-f(x)$

  \end{center}
\end{tcolorbox}%}}}

\section{Difference Table}%{{{
When we arrange the difference of various order of a function systematically 
along with arguments and entries in a table then it is known as difference
table.

Let $x_0, x_1, x_2, x_3$ are the four equidistant values of $x$ such that $x_1-
x_0=x_2-x_1=x_3-x_2=h$ and $f(x_0), f(x_1), f(x_2), f(x_3),$ are the
corresponding values of $f(x)$ then the difference table is

\begin{table}[ht]
\resizebox{\columnwidth}{!}{
  \LARGE
  \begin{tabular}{||c|c|c|c||}
    \hline
    $xf(x)$ & $\Delta f(x)$ & $\Delta^2f(x)$ & $\Delta^3f(x)$ \\
    \hline
    \hline
    $x_0f(x_0)$ &
      $\Delta f(x_0) = f(x_1)-f(x_0)$ &
      \multirow{2}{*}{$\Delta^2f(x_0)=\Delta f(x_1)-\Delta f(x_0)$} &
      \multirow{4}{*}{$\Delta^3f(x_0)=\Delta^2f(x_1)-\Delta^2f(x_0)$}\\
      \cline{1-2}
    $x_1f(x_1)$ &
      $\Delta f(x_1) = f(x_2)-f(x_1)$ & & \\
      \cline{1-3}
    $x_2f(x_2)$ &
      $\Delta f(x_2) = f(x_3)-f(x_2)$ &
      \multirow{2}{*}{$\Delta^2f(x_1)=\Delta f(x_2)-\Delta f(x_1)$} & \\
      \cline{1-2}
    $x_3f(x_3)$ &
      $\Delta f(x_3) = f(x_4)-f(x_3)$ & & \\
      \hline
  \end{tabular} }
\end{table}%}}}

\section{Interpolation}%{{{
Interpolation is the technique of estimating approximately the value of the
dependent variable corresponding to a value of the independent variable, 
between its two extreme value of the independent and dependent variable.

\subsection{Assumptions of interpolation}
The application of the principal of interpolation is subject of the following
assumptions:
\begin{enumerate}
  \item There is no certain jumps in figure from one period to another period.
  \item The rate of change of figures is uniform overtime.
\end{enumerate}

\section{Different interpolation formula}
\subsection{Newton's forwards interpolation}
It is to be used to estimate the value of the depended variable corresponding 
to a given value of the independent variable.
\begin{enumerate}
  \item If the independent variable $x$ increases by equal intervals.
  \item The value of $x$ corresponding to which the value of $y$ is to be 
    interpolated in the 1\textsuperscript{st} hub of the series.
\end{enumerate}

The formula is 
\begin{align*}
  f(x)=f(x_0)+U\Delta f(x_0) &+ \frac{U(U-1)}{2!}\Delta^2f(x_0)+\frac{U(U-1)
  (U-2)}{3!}\Delta^3f(x_0)\\
                             &+ \dots+\frac{U(U-1)(U-2)\dots\{U-(n-1)\}}{n!}
                             \Delta^nf(x_0)\\
                             \\
                             &\text{where }U=(\frac{x-x_0}{h})\text{and }h
                             \text{ is the interval of differencing.}
\end{align*}

%{{{
\begin{tcolorbox}
  \centering
  \Large
  \bfseries
  Derivation
\end{tcolorbox}
Let, $y=f(x)$, be a function,

Here, the argument's $x_0,x_1,\dots.x_n$ are equidistant. $x_1-x_0=x_2-x_1=\dots=x_n-x_{n-1}=h$

Let us take a polynomial of $n^{th}$ degree as our interpolation formula.

Let 
\begin{equation}\label{dv1}
  f(x)=a_0+a_1(x-x_0)+a_2(x-x_0)(x-x_1)+\dots+a_3(x-x_0)(x-x_1)\dots(x-x_{n-1})
\end{equation}
where $a_0,a_1,\dots,a_n$ are the constants to be determined.

Now, putting $x=x_0$ in equation number (\ref{dv1}) 
\begin{align*}
  f(x_0)&=a_0
\end{align*}
Again, putting $x=x_1$ in equation (\ref{dv1})
\begin{align*}
  f(x_1)&=a_0+a_1(x_1-x_0)\\
  \Rightarrow f(x_1)&=f(x_0)+a_1*h\\
  \Rightarrow a_1 &=\frac{f(x_1)-f(x_0)}{h}\\
  &=\frac{\Delta f(x_0)}{h}
\end{align*}

Again, putting $x=x_2$ in equation (\ref{dv1})
\begin{align*}
              f(x_2)  &= a_0+a_1(x_2-x_0)+a_2(x_2-x_0)(x_2-x_1)\\
  \Rightarrow f(x_2)  &= f(x_0)+\frac{\Delta f(x_0)}{h}(x_2-x_1+x_1-x_0)\\ &+ a_2(x_2-x_1+x_1-x_0)(x_2-x_1)\\
  \Rightarrow f(x_2)  &= f(x_0)+\frac{\Delta f(x_0)}{h}*2h+a_2*2h*h\\
  \Rightarrow f(x_2)  &= f(x_0)+2\Delta f(x_0)+2h^2a_2\\
  \Rightarrow a_2     &= \frac{f(x_2)-2\Delta f(x_0)-f(x_0)}{2h^2}\\
  \Rightarrow a_2     &= \frac{f(x_2)-2f(x_1)+2f(x_0)-f(x_0)}{2h^2}\\
                      &= \frac{f(x_2)-2f(x_1)+f(x_0)}{2!h^2}\\
                      &= \frac{\Delta^2 f(x_0)}{2!h^2}
\end{align*}

Again, putting $x=x_3$ is equation (\ref{dv1})
\begin{equation*}
  a_3=\frac{\Delta^3f(x_0)}{3!h^3}\\
\end{equation*}

....................

....................................


Similarly, putting $x=x_n$ in equation (\ref{dv1})
\begin{equation*}
  a_n=\frac{\Delta^nf(x_0)}{n!h^n}
\end{equation*}

Now putting the value of $a_0,a_1,a_2,a_3,\dots,a_n$ in equation (\ref{dv1}), we get,
\begin{equation*}\label{dv2}
  f(x)=f(x_0)+\frac{\Delta f(x_0)}{h}(x-x_0)+\frac{\Delta^2f(x_0)}{2!h^2}(x-x_0)(x-x_1)
\end{equation*}
\begin{equation*}\label{dv2}
      +\frac{\Delta^3f(x_0)}{3!h^3}(x-x_0)(x-x_1)(x-x_2)
\end{equation*}
\begin{equation}\label{dv2}
      +\dots+\frac{\Delta^nf(x_0)}{n!h^n}(x-x_0)(x-x_1)\dots(x-x_{n-1})
\end{equation}

Let us put $U=\frac{x-x_0}{h}$ in equation (\ref{dv2})
\begin{align*}
  f(x)&=f(x_0)+U\Delta f(x_0)+\frac{U(U-1)}{2!}\Delta^2f(x_0)+\frac{U(U-1)(U-2)}{3!}\Delta^3f(x_0)\\
  &+\dots+\frac{U(U-1)\dots\{U-(n-1)\}}{n!}\Delta^nf(x_0)
\end{align*}
%}}}

\subsection{Newton's backwards interpolation formula}
Let, $f(x)$ be a polynomial  of degree $n$ and $n+1$ equidistant values of $x$
along then
corresponding values of $f(x)$. The Newton's backward interpolation formula is

\begin{align*}
  f(x) &= f(x_n)+U\nabla f(x_n-1)+\frac{U(U+1)}{2!}\nabla^2f(x_n-2)+\frac{U(U+1)(U+2)}{3!}\nabla^3f(x_n-3)\\
       &+\dots+\frac{U(U+1)\dots\{U+(n-1)\}}{n!}\nabla^nf(x_0)
\end{align*}

where, $U=\frac{(x-x_n)}{h}$, where h is the interval of differences.

The Newton's backward interpolation formula is used to estimate values of the
dependent 
variable corresponding to a given value of the independent variable.

\begin{enumerate}
  \item The independent variable $x$ increases by equal intervals.
  \item The value of $x$ corresponding to each value of $y$ is to be
    interpolated in the $2^{nd}$ half of the series.
\end{enumerate}

\subsection{Lagrenge's interpolation formula}
Let, $f(x_0),f(x_1),\dots,f(x_n)$ be the entries corresponding to the argument
$x_0,x_1,\dots,x_n$ which are not necessarily in equal intervals. The
interpolation formula given by Lagrenge to estimate the value of $y$
corresponding to a value of $x$ between any two consecutive values of the
given values is

\begin{align*}
  f(x)&=\frac{(x-x_1)(x-x_2)\dots(x-x_n)}{(x_0-x_1)(x_0-x_2)\dots(x_0-x_n)}f(x_0)+\frac{(x-x_0)(x-x_2)\dots(x-x_n)}{(x_1-x_0)(x_1-x_2)\dots(x_1-x_n)}f(x_1)\\
  &+ \dots +\frac{(x-x_1)(x-x_2)\dots(x-x_{n-1})}{(x_n-x_0)(x_n-x_1)\dots(x_n-x_{n-1})}f(x_n)
\end{align*}

Lagrange's Interpolation formula is usually and when the values of the
independent variable $x$ are not equidistant. To apply Lagrange's formula the
value of $x$ corresponding to which the value of $y$ is to be interpolated may
be anywhere in between the first and last terms.

\vskip10pt
\emph{To apply Lagrange's formula, constuction of difference table is not
needed.}%}}}

\section{Numerical Integration}%{{{
Numerical Integration is the process of finding or evaluating definate
integral is 

$I=\int_{a}^{b}f(x)dx$ from a set of numerical value of the integrant $f(x)$.
If it is applied to the integration of a function of a single variable then
the process is known as quadrature. The problem of numerical integration is
solved by first approximating the integrant by a polynomial with the help of
an interpolation formula and then integrating this expression between the
desired limits.

\begin{align*}
  \int xdx&=\frac{x^2}{2}\\
  \int x^ndx&=\frac{x^{n+1}}{n+1}
\end{align*}

\subsection{General quadrature formula}
Let $y=f(x)$ be the given integrant and the corresponding integral is 

\begin{equation*}
  I=\int_{a=x_0}^{b=x_0+nh}f(x)dx
\end{equation*}

Let us suppose that we are given a set of numerical values of the integrant
corresponding to some equidistant values of $x$.

Let us divide the range $(a,b)$ in $n$ equal parts, each of which with is $nh$
i.e. $b-a=nh$.

Say $a=x_0, a+h=x_0+h,\dots a+nh=x_0+nh$

Let us take the Newton's Forward Interpolation Formula as an approximating of $f(x)$.

\begin{align*}
  \because I=\int_{a=x_0}^{b=x_0+nh}f(x)dx\\
\end{align*}

\begin{align*}
=\int_{x_0}^{x_0+nh}f(x_0)+U\Delta f(x_0) &+\frac{U(U-1\Delta^2f(x_0))}{2!}+\frac{U(U-1)(U-2)}{3!}\Delta^3f(x)\\
& \dots\frac{U(U-1)(U-2)\dots\{U-(n-1)\}}{n!}\Delta^nf(x_0)
\end{align*}

\begin{align*}
=\int_{x_0}^{x_0+nh}y_0+\Delta y_0+\frac{U(U-1)}{2!}\Delta^2y_0 &+ \frac{U(U-1)(U-2)}{3!}\Delta^3y_0\\
&+ \dots+\frac{U(U-1)\dots\{U-(n-1)\}}{n!}\Delta^ny_0\}dx
\end{align*}

where,

\begin{align*}
              U &= \frac{x-x_0}{h}\\
  \Rightarrow x &= x_0+hu\\
  \Rightarrow \frac{dx}{du} &= 0+h\frac{d(u)}{du}=h\\
  \Rightarrow dx &= hdu
\end{align*}

when, $x=x_0$, $U=\frac{x_0-x_0}{h}=0$

where, $x=(x_0+nh)$, $U=\frac{x_0+nh-x_0}{h}=\frac{nh}{h}=n$

\begin{align*}
  I &= \int_{0}^{n}\{y_0+U\Delta y_0+\frac{U^2-U}{2!}\Delta^2y_0+\frac{U^3-3U^2+2U}{3!}\Delta^3y_0\\
    &+ \dots +\ upto\ (n+1)\ terms\} hdu\\
    \ \\
    &= h[y_0U+\Delta y_0\frac{U^2}{2}+\frac{\Delta^2y_0}{2!}(\frac{U^3}{3}-\frac{U^2}{2})+\frac{\Delta^3y_0}{3!}(\frac{U^4}{4}-\frac{U^3}{3}+\frac{U^2}{2})\\
    &+ \dots+upto\ (n+1)\ terms]_{0}^{n}\\
    \ \\
    &= h[ny_0+\frac{n^2}{2}\Delta y_0+(\frac{n^3}{3}-\frac{n^2}{2})\frac{\Delta^2y_0}{2!}+(\frac{n^4}{4}-n^3-n^2)\frac{\Delta^3y_0}{3!}\\
    &+ \dots+upto\ (n+1)\ terms]
\end{align*}

This is called the general quadrature formula.

\subsection{Trapezoidal Rule}
The general quadrature formula is

\begin{align*}
  I &=\int_{a=x_0}^{b=x_0+nh}f(x)dx\\
    &= h[ny_0+\frac{n^2}{2}\Delta y_0+(\frac{n^3}{3}-\frac{n^2}{2})\frac{\Delta^2y_0}{2!}+(\frac{n^4}{4}-n^3+n^2)\frac{\Delta^3y_0}{3!}
\end{align*}

\begin{align}\label{dv3}
    &+ \dots+upto\ (n+1)\ terms]
\end{align}

Putting $n=1$ in equation number \ref{dv3}, and neglecting all the differece 
higher than 1, we get:

\begin{align*}
  I_1 &= \int_{x_0}^{x_0+h}f(x)dx\\
      &= h[y_0+\frac{1}{2}\Delta y_0]\\
      &= h[y_0+\frac{1}{2}(y_1-y_0)]\\
      &= h[\frac{2y_0+y_1-y_0}{2}]\\
      &= h[\frac{y_0+y_1}{2}]
\end{align*}

Similarly,

\begin{align*}
  I_2 &= \int_{x_0+h}^{x_0+2h}f(x)dx = h[\frac{y_1+y_2}{2}]
\end{align*}

\begin{align*}
  I_3 &= \int_{x_0+2h}^{x_0+3h}f(x)dx = h[\frac{y_2+y_3}{2}]
\end{align*}

\dots 

\dots 

\begin{align*}
  I_n &= \int_{x_0+(n-1)h}^{x_0+nh}f(x)dx = h[\frac{y_{(n-1)}+y_n}{2}]
\end{align*}

Adding these n integrals we get

\begin{align*}
  \int_{x_0}^{x_0+nh}f(x)dx &= h[(\frac{y_0-y_1}{2})+(\frac{y_1-y_2}{2})+\dots+(\frac{y_{(n-1)}+y_n}{2})]\\
                            &= h[(\frac{y_0+y_n}{2})+(y_1+y_2+y_3+\dots+y_{n-1})]
\end{align*}

This is called the Trapezoidal Rule.

\textbf{Condition for validity of trapezoidal rule}
\begin{enumerate}
  \item $f(x)$ should be a polynomial of degree of 1, i.e. a 
    straight line of the form $f(x)=a+bx$
  \item The value of $x$ should be equidistant
  \item The number of divisions of the integral $(x_0,x_0+nh)$ 
    should be multiple of 1, like 3, 5, 7, etc.
\end{enumerate}

\subsection{Simpson's $\frac{1}{3}^{rd}$ rule}
The general quadrature formula is 

\begin{align*}
  I &=\int_{a=x_0}^{b=x_0+nh}f(x)dx\\
    &= h[ny_0+\frac{n^2}{2}\Delta y_0+(\frac{n^3}{3}-\frac{n^2}{2})\frac{\Delta^2y_0}{2!}+(\frac{n^4}{4}-n^3+n^2)\frac{\Delta^3y_0}{3!}\\
    &+ \dots+upto\ (n+1)\ terms]
\end{align*}

Putting $n=2$ and neglecting all the differences higher then second order.

\begin{align*}
  I_1=\int_{a=x_0}^{b=x_0+2h}f(x)dx &= h[2y_0+2\Delta y_0+(\frac{8}{3}-2)\frac{\Delta^2y_0}{2!}]\\
                                    &= h[2y_0+2(y_1-y_0)+(\frac{8}{3}-2)\frac{y_2-2y_1+y_0}{2}]\\
                                    &= h[2y_0+2y_1-2y_0+\frac{2}{3}*\frac{y_2-2y_1+y_0}{2}]\\
                                    &= h[2y_1+\frac{y_2-2y_1+y_0}{3}]\\
                                    &= h[\frac{6y_1+y_2-2y_1+y_0}{3}]\\
                                    &= h[\frac{y_0+4y_1+y2}{3}]\\
                                    &= \frac{h}{3}(y_0+4y_1+y_2)
\end{align*}

Simillarly,

\begin{align*}
  I_4=\int_{x_0+2h}^{x_0+4h}f(x)dx &= \frac{h}{3}(y_2+4y_3+y_4)
\end{align*}

And,

\begin{align*}
  I_n=\int_{x_0+(n-2)h}^{x_0+nh}f(x)dx &= \frac{h}{3}(y_{n-2}+4y_{n-1}+y_n)
\end{align*}

On adding all these integrals, we get:

\begin{align*}
  \int_{x_0}^{x_0+nh}f(x)dx &= \frac{h}{3}(y_0+y_n)+4(y_1+y_3+\dots+y_{n-1})+2(y_2+y_4+\dots+y_{n-2})
\end{align*}

This formula is known as Simpson's $\frac{1}{3}^{rd}$ formula.

\textbf{Condition for validity of Simpson's $\frac{1}{3}^{rd}$ rule.}

\begin{enumerate}
  \item $f(x)$ should be a polynomial of degree 2 of the form $f(x)=a+bx+cx^2$
  \item The values of x should be equidistant
  \item The number of division of the interval $(x_0,x_0+nh)$ should be multiple of 2 like 2, 4, 6, 8, 10, etc.
\end{enumerate}

\subsection{Simpson's $\frac{3}{8}^{th}$ rule}
The general quadrature formula is 

\begin{align*}
  I &=\int_{a=x_0}^{b=x_0+nh}f(x)dx\\
    &= h[ny_0+\frac{n^2}{2}\Delta y_0+(\frac{n^3}{3}-\frac{n^2}{2})\frac{\Delta^2y_0}{2!}+(\frac{n^4}{4}-n^3+n^2)\frac{\Delta^3y_0}{3!}\\
    &+ \dots+upto\ (n+1)\ terms]
\end{align*}

Putting $n=3$ and neglicting all the difference higher than $3^{rd}$ order.

\begin{align*}
  I_1=\int_{x_0}^{x_0+3h}f(x)dx &= h[3y_0+\frac{3}{2}\Delta y_0+(\frac{27}{3}-\frac{9}{2})\frac{\Delta^2y_0}{2!}+(\frac{81}{4}-27+9)\frac{\Delta^3y_0}{3!}]\\
                          &= h[3y_0+\frac{3}{2}(y_1-y_0)+(9-\frac{9}{2})(\frac{y_2-2y_1+y_0}{2})+(\frac{81}{4}-27+9)\\
                          &  (\frac{y_3-3y_2+3y_1-y_0}{6})]\\
                          \ \\ 
                          &= h[3y_0+\frac{9}{2}(y_1-y_0)+\frac{9}{3}(\frac{y_2-2y_1+y_0}{2})+\frac{3}{8}(y_3-3_2+3y_1-y_0)]\\
                          &= \frac{3h}{8}[8y_0+12(y_1-y_0)+6(y_2-2y_1+y_0)+(y_3-3y_2+3y_1-y_0)]\\
                          &= \frac{3h}{8}[y_0+3y_1+3y_2+y_3]\\
\end{align*}

Similarly,

\begin{align*}
  \int_{x_0+3h}^{x_0+6h}f(x)dx &= \frac{3h}{8}[y_3+3y_4+3y_5+y_6]\\
\end{align*}
\dots

\dots
\begin{align*}
  \int_{x_0+(n-3)h}^{x_0+nh}f(x)dx &= \frac{3h}{8}[y_{n-3}+3y_{n-2}+3y_{n-1}+y_n]\\
\end{align*}

Adding all the integrals,

\begin{align*}
  \int_{x_0}^{x_0+nh}f(x)dx=\frac{3h}{8}[(y_0+y_n) &+ 3(y_1+y_2+y_4+y_5+\dots+y_{n-1})\\
                                                   &+ 2(y_3+y_6+\dots+y_{n-3})]
\end{align*}

This formula is known as Simpson's $\frac{3}{8}^{rd}$ formula.

\textbf{Conditions for validity of Simpson's $\frac{3}{8}^{rd}$ rule.}

\begin{enumerate}
  \item $f(x)$ should be a polynomial of degree 3 of the form 
    $f(x)=a+6x+cx^2+dx^3$
  \item The value of $x$ should be equidistant.
  \item The number of division of the interval $(x_0,x_0+nh)$
    should be multiple of 3 like 3, 6, 9, etc.
\end{enumerate}%}}}

\section{Algebraic and transcendental equation\todo{Sir or notes tika lekhsi}}
By the help of algebraic approach we can solve equations of all degrees upto and including the 4\textsuperscript{th} degree and it is also known how to compute the rules of numerical equation of any degree. Algebra is silent however, on the solution of such type of equations as $ax+b\log x+c=0$ or $ae^x+b\tan x=c$ etc. These are called transcendental equations and no general method exists for finding their roots in terms of their coefficient. When the coefficient of such equations are pure numbers, it is always possible to compute the roots to any degree of accuracy.

\subsection{Bisection method}
This method of solving a transcendental equation consists in locating the roots of the equation $f(x)=0$ between two numbers, say $x_0$ and $x_1$, such that $f(x)$ is continuous for $x_0\le x \le x_1$ and $f(x_1)$ are the opposite signs so that $f(x_0)\cdot f(x_1) < 0$ IE, the curve crosses the
x-axis between $x_0$ and $x_1$ and the desire root of the given equation is approximately,

\begin{align*}
  x_2 &= \frac{x_0+x_1}{2}\\
  x_3 &= \frac{x_0+x_2}{2}\text{, provided }f(x_0)\cdot f(x_2)<0\\
      &= \frac{x_1+x_2}{2}\text{, provided }f(x_1)\cdot f(x_2)<0\\
  \text{Further } x_4 &= \frac{x_1+x_2}{2}\text{, provided }f(x_0)\cdot f(x_3)<0\\
  &\text{and so on\dots}
\end{align*}

Thus on each iteration, we either find the root with desired accuracy or we narrow down the range to half of the previous interval.

\subsection{Newton-Raphson method}
When the derivative of $f(x)$ is a simple expression and easily found, the real root of $f(x)=0$ can be computed rapidly by a process called the Newton-Raphson method. The underlying idea of the method is due to Newton, but the method as now used is due to Raphson.

To derive a formula for computing real roots by this method, let `$a$' denote an approximate value of the desire root and also let `$h$' is the correlation which is applied to `$a$' to get the exact value of the root of the given equation so that $x=a+h$.

The equation $f(x)=0$, then $f(a+h)=0$

Expanding by Taylor's theorem, we get,

\begin{align*}
  f(a+h) &= f(a)+hf'(a)+\frac{h^2}{2}f''(a+\theta h), \ \ \ \ 0\le\theta\le 1\\
         &= 0\\
  \therefore f(a)&+hf'(a)+\frac{h^2}{2}f''(a+\theta h) = 0
\end{align*}

Now if $h$ is very very small, we can neglect the term containing $h^2$ and as such

\begin{align*}
  f(a)+hf'(a) &= 0\\
\end{align*}
\vskip-40pt
\begin{align}\label{dv4}
  \Rightarrow h=\frac{-f(a)}{f'(a)} &= h_1\text{ (say) }
\end{align}

The improved value of the root is then
$$
a_1=a+h_1=a-\frac{f(a)}{f'(a)}
$$

The succeeding approximation are 
\begin{align*}
  a_2=a_1+h_2 &= a_1-\frac{f(a_1)}{f'(a_1)}\\
  \text{Similarly, } a_3=a_2+h_3 &= a_2-\frac{f(a_2)}{f'(a_2)}\\
  a_{n}=a_{n-1}+h_n &= a_{n-1}-\frac{f(a_{n-1})}{f'(a_{n-1})}\\
\end{align*}

Equation \ref{dv4} is the fundamental formula in the Newton-Raphson method. It is evident from this formula that larger the derivative $f'(x)$, the small is the correlation which must be applied to get the correct root of the equation. This means that when the graph is nearly vertical, where it crosses the x-axis, the correct value of the root can be found with great rapidity and very little labor. On the other hand, when the graph is horizontal on the x-axis, the Newton-Raphson method will be a slow process for computing the real root of the given equation or might even fail.

The Newton-Raphson method should never be used when the graph of $f(x)$ is nearly horizontal where it crosses the x-axis.

\section{Method of false position/Regular falsi method}
The oldest method for computing the real roots of a numerical equation is the method of false position. In this method, we find two numbers $x_1$ and $x_2$ between which the root lies. This numbers should be as close as possible. Since the root lies between $x_1$ and $x_2$, the graph $y=f(x)$ must
cross the x-axis between $x=x_1$

\end{document}
