\documentclass[10pt, twoside, a4paper]{article}

% Packages{{{
\usepackage[sc]{mathpazo}
\usepackage{graphicx}
  \graphicspath{ {./images/} }
\usepackage[T1]{fontenc}
  \linespread{1.1}
\usepackage{microtype}
\usepackage{soul}
\usepackage{tcolorbox}
\usepackage{amsmath}
\usepackage{amssymb}

\usepackage[english]{babel}

\usepackage[hmarginratio=1:1,top=32mm,columnsep=15pt]{geometry}
\usepackage[hang, small,labelfont=bf,up,textfont=it,up]{caption}
\usepackage{booktabs}

\usepackage{lettrine}

\usepackage{enumitem}
  \setlist[itemize]{noitemsep}

\usepackage{titlesec}
% \renewcommand\thesection{Roman{section}}
% \renewcommand\thesubsection{{subsection}}
\titleformat{\section}[block]{\Large\scshape\centering}{\thesection.}{1em}{}
\titleformat{\subsection}[block]{\large}{\thesubsection.}{1em}{}

\usepackage{fancyhdr} % Headers and footers
\pagestyle{fancy} % All pages have headers and footers
\fancyhead{} % Blank out the default header
\fancyfoot{} % Blank out the default footer
\renewcommand{\headrulewidth}{0pt}
\fancyhead[RE,LO]{\thepage} % Custom footer text

\usepackage{titling}

\usepackage{hyperref}
%}}}

% Title{{{
\setlength{\droptitle}{-4\baselineskip}

\pretitle{\begin{center}\Huge\bfseries}
\posttitle{\end{center}}
\title{Numerial and statistical method}
\author{ by
  \textsc{Swapnil}%\\[1px]
}
\date{}
%}}}

\begin{document}
% FontMatter {{{
\maketitle

% \twocolumn
%}}}
% \parindent0pt

\section{Calculas of finite difference}%{{{
The calculas of finite difference deals with the changes in the values of
the dependent variable due to the changes in the independent variable.

Let, `$y$' be a function of `$x$' represented by $y=f(x)$. Let, $a, a+h, a+2h,
\dots,(a+nh)$ be the $(n+1)$ equidistant values of $x$ and $f(a), f(a+h),\dots
f(a+nh)$ be the corresponding values of $f(x)$ are called entries.

\vskip10pt
\noindent Again,
\begin{align*}
  (a+h)-a &= (a+2h)-(a+h) =(a+3h)-(a+2h)=\dots\\
          &= (a+nh)-\{a+(n-1)h\}\\
          &= h \text{\hskip10pt [where `h' is called the interval of differences]}
\end{align*}%}}}

\subsection{Difference of first order ($\Delta$)}%{{{
\begin{equation*}
  \Delta f(x)=f(x+h)-f(x)
\end{equation*}%}}}

\subsection{Difference of second order ($\Delta^2$)}%{{{
\begin{align*}
  \Delta^2 f(x) &= \Delta \{\Delta f(x)\}\\
                &= \Delta \{f(x+h)-f(x)\}\\
                &= \Delta f(x+h)-\Delta f(x)\\
                &= f(x+2h)-f(x+h)-\{f(x+h)-f(x)\}\\
   \Delta^2f(x) &= f(x+2h)-2f(x+h)+f(x)\\
\end{align*}%}}}

\subsection{Difference of third order ($\Delta^3$)}%{{{
\begin{align*}
  \Delta^3 f(x) &= \Delta \{\Delta^2 f(x)\}\\
                &= \Delta \{f(x+2h)-2f(x+h)+f(x)\}\\
                &= f(x+3h)-f(x+2h)-2f(x+2h)+2f(x+h)+f(x+h)-f(x)\\
                &= f(x+3h)-3f(x+2h)+2f(x+h)-f(x)
\end{align*}%}}}

\section{Difference Table}%{{{
When we arrange the difference of various orders of a function systematically 
along with arguments and entries in a table, then it is known as difference 
table.
% Let, $x_0,x_1,x_2,x_3$ are 4 equidistant values of x such that
% $x_1-x_0,x_2-x_1,x_3-x_2=h$ and $f(x_0),f(x_1),f(x_2),f(x_3)$
% are the corresponding values of $f(x)$, then the difference
% table is
%}}}

\section{Interpolation}%{{{
Interpolation is a technique of estimating the approximates of the value of
the value of the independent variable, between its two extreme values.

\subsection{Assumption of interpolation}
The application of principal of interpolation is subject to the following
assumption.

\begin{itemize}
  \item There is no sudden jumps in figures from one period to another period.
  \item The rate of change of figure is uniform, over-time.
\end{itemize}

\subsection{Different Interpolation Formula}%{{{
\subsubsection{Newton's Forward Interpolation Formula}%{{{
It is used to estimate the value of the dependent variable corresponding to a 
given value of the independent variable, if:

\begin{itemize}
  \item The independent variable $x$, increases by equal intervals.
  \item The value of $x$, corresponding to which the value of $y$ is to be 
    interpolated in the first hub of the series.
\end{itemize}
The formula is

\begin{tcolorbox}
\begin{align*}
  f(x)=f(x_0)+U\Delta f(x_0) &+ \frac{U(U-1)}{2!}\Delta^2f(x_0)
       +\frac{U(U-1)(U-2)}{3!}\Delta^3f(x_0)\\
                             &+ \dots+\frac{U(U-1)(U-2)\dots\{U-(n-1)\}}{n!}
                                \Delta^nf(x_0)
\end{align*}
\end{tcolorbox}

\begin{equation*}
  \text{where, } U = \frac{x-x_0}{h} \text{ and } h \text{ is the interval of
  differences.}
\end{equation*}
%}}}

\subsubsection{Newton's Backward Interpolation Formula}%{{{
Let, $f(x)$ be a polynomial of degree $n$ and $n+1$ equidistant values of $x$ 
along then corresponding values of $f(x)$. The Newton's backward interpolation
formula is

\begin{tcolorbox}
\begin{align*}
  f(x) = f(x_n)+U\nabla f(x_n-1) &+ \frac{U(U+1)}{2!}\nabla^2f(x_n-2)
  +\frac{U(U+1)(U+2)}{3!}\nabla^3f(x_n-3)\\
                                 &+ \dots+\frac{U(U+1)\dots\{U+(n-1)\}}{n!}
                                 \nabla^nf(x_0)
\end{align*}
\end{tcolorbox}

\begin{equation*}
  \text{where, } U = \frac{x-x_0}{h} \text{ and } h \text{ is the interval of
  differences.}
\end{equation*}

The Newton's backward interpolation formula is used to estimate values of the 
dependent variable corresponding to a given value of the independent variable.

\begin{itemize}
  \item The independent variable $x$ increases by equal intervals.
  \item The value of $x$ corresponding to each value of $y$ is to be
    interpolated in the 2\textsuperscript{nd} half of the series.
\end{itemize}
%}}}

\subsubsection{Lagrange's Interpolation Formula}%{{{
Let, $f(x_0),f(x_1),\dots,f(x_n)$ be the entries corresponding to the
arguments $x_0, x_1,\dots,x_n$ which are not necessarily in equal interval.
The interpolation formula given by Lagrange to estimate the value of $y$
corresponding to a value of $x$ between any two consecutive values of the
given values is
\begin{tcolorbox}
  \begin{align*}
    f(x) &= \frac{(x-x_1)(x-x_2)
             \dots(x-x_n)}{(x_0-x_1)(x_0-x_2)
             \dots(x_0-x_n)}f(x_0)
          + \frac{(x-x_0)(x-x_2)
             \dots(x-x_n)}{(x_1-x_0)(x_1-x_2)
             \dots(x_1-x_n)}f(x_1)\\
         &+ \dots 
          + \frac{(x-x_1)(x-x_2)
             \dots(x-x_{n-1})}{(x_n-x_0)(x_n-x_1)
             \dots(x_n-x_{n-1})}f(x_n)
  \end{align*}
\end{tcolorbox}

Lagrange's interpolation formula is usually and when the values of the
independent values of the independent variable $x$ are not equidistant. To 
apply Lagrange's formula the value of $x$ corresponding to which the value of
$y$ us to be interpolated may be anywhere in between the first and last terms.
No difference table is needed to apply Lagrange's formula.%}}}
%}}}

%}}}

\section{Numerical Integration}%{{{
\begin{center}
  \emph{Numerical Integration is the process of finding or evaluating definate
  integrals.}
\end{center}

$I=\int_{a}^bf(x)dx$ from a set of numerical value of the integrant $f(x)$. If 
it is applied to the integration of a function of a single variable then the
process is known as quadrature. The problem of numerical integration is solved 
by first apporximating the integrant by a polynomial with the help of an
interpolation formula and then integrating this expression between the desired
limits.
\begin{align*}
  \int xdx   &= \frac{x^2}{2}\\
  \int x^ndx &= \frac{x^{n+1}}{n+1}
\end{align*}

\subsection{General quadrature formula}%{{{
\begin{tcolorbox}
\begin{align*}
  I &= \int_{0}^{n}\{y_0+U\Delta y_0
     +\frac{U^2-U}{2!}\Delta^2y_0+\frac{U^3-3U^2+2U}{3!}\Delta^3y_0\\
    &+ \dots +\ upto\ (n+1)\ terms\} hdu\\
    \ \\
    &= h[y_0U+\Delta y_0\frac{U^2}{2}
     +\frac{\Delta^2y_0}{2!}(\frac{U^3}{3}-\frac{U^2}{2})
     +\frac{\Delta^3y_0}{3!}(\frac{U^4}{4}-\frac{U^3}{3}+\frac{U^2}{2})\\
    &+ \dots+upto\ (n+1)\ terms]_{0}^{n}\\
    \ \\
    &= h[ny_0+\frac{n^2}{2}\Delta y_0
     +(\frac{n^3}{3}-\frac{n^2}{2})\frac{\Delta^2y_0}{2!}
     +(\frac{n^4}{4}-n^3-n^2)\frac{\Delta^3y_0}{3!}\\
    &+ \dots+upto\ (n+1)\ terms]
\end{align*}
  
\end{tcolorbox}
%}}}

\subsection{Trapezoidal Rule}%{{{
The general quadrature formula is
\begin{align*}
I &=\int_{a=x_0}^{b=x_0+nh}f(x)dx\\
  &= h[ny_0+\frac{n^2}{2}\Delta y_0
   + (\frac{n^3}{3}-\frac{n^2}{2})
      \frac{\Delta^2y_0}{2!}
   + (\frac{n^4}{4}-n^3+n^2)
      \frac{\Delta^3y_0}{3!}
    &+ \dots+upto\ (n+1)\ terms]
\end{align*}
Putting $n=1$ in the above equation, and neglecting all the differences higher
than first order, we get:
\begin{align*}
  I_1 &= \int_{x_0}^{x_0+h}f(x)dx\\
      &= h[y_0+\frac{1}{2}\Delta y_0]\\
      &= h[y_0+\frac{1}{2}(y_1-y_0)]\\
      &= h[\frac{2y_0+y_1-y_0}{2}]\\
      &= h[\frac{y_0+y_1}{2}]\\
  \text{Similarly,}\\
  I_2 &= \int_{x_0+h}^{x_0+2h}f(x)dx = h[\frac{y_1+y_2}{2}]\\
  I_3 &= \int_{x_0+2h}^{x_0+3h}f(x)dx = h[\frac{y_2+y_3}{2}]\\
\dots \\
  I_n &= \int_{x_0+(n-1)h}^{x_0+nh}f(x)dx = h[\frac{y_{(n-1)}+y_n}{2}]\\
  \text{Adding these n integrals, we get}\\
\end{align*}
\begin{tcolorbox}
\begin{align*}
  \int_{x_0}^{x_0+nh}f(x)dx &= h[(\frac{y_0-y_1}{2})+(\frac{y_1-y_2}{2})
                             + \dots+(\frac{y_{(n-1)}+y_n}{2})]\\
                            &= h[(\frac{y_0+y_n}{2})+(y_1+y_2+y_3
                             + \dots+y_{n-1})]\\
  \text{This is called the Trapezoidal rule.}
\end{align*}
\end{tcolorbox}

\subsubsection{Condition for validity of trapezoidal rule}
\begin{itemize}
  \item $f(x)$ should be a polynomial of degree of 1, IE a straight line of the
    form $f(x)=a+bx$
  \item The value of $x$ should be equidistant
  \item The number of divisions of ther integral ($x_0,x_0+nh$) should be
    multiple of 1, like 3, 5, 7, etc
\end{itemize}
%}}}

\subsection{Simpson's $\frac{1}{3}^{rd}$ rule}%{{{
Putting $n=2$ in the general quadrature formula, and neglecting all the
differences higher then second order, we get:

\begin{align*}
  I_1=\int_{a=x_0}^{b=x_0+2h}f(x)dx &= h[2y_0+2\Delta y_0
                                      +(\frac{8}{3}-2)\frac{\Delta^2y_0}{2!}]\\
                                    &= h[2y_0+2(y_1-y_0)+(\frac{8}{3}-2)
                                       \frac{y_2-2y_1+y_0}{2}]\\
                                    &= h[2y_0+2y_1-2y_0+\frac{2}{3}
                                      *\frac{y_2-2y_1+y_0}{2}]\\
                                    &= h[2y_1+\frac{y_2-2y_1+y_0}{3}]\\
                                    &= h[\frac{6y_1+y_2-2y_1+y_0}{3}]\\
                                    &= h[\frac{y_0+4y_1+y2}{3}]\\
                                    &= \frac{h}{3}(y_0+4y_1+y_2)\\
  \text{Simillarly,}\\
   I_4 &= \int_{x_0+2h}^{x_0+4h}f(x)dx = \frac{h}{3}(y_2+4y_3+y_4)\\
  \text{And,}\\
  I_n &= \int_{x_0+(n-2)h}^{x_0+nh}f(x)dx = \frac{h}{3}(y_{n-2}+4y_{n-1}+y_n)\\
  \text{Adding these integrals, we get}\\
\end{align*}
\begin{tcolorbox}
\begin{align*}
  \int_{x_0}^{x_0+nh}f(x)dx &= \frac{h}{3}(y_0+y_n)+4(y_1+y_3+
                               \dots+y_{n-1})\\
                            &+ 2(y_2+y_4+\dots+y_{n-2})\\
  \text{This is Simpson's $\frac{1}{3}^{rd}$ formula.}
\end{align*}
\end{tcolorbox}
\subsubsection{Condition for validity of Simpson's $\frac{1}{3}^{rd}$ rule.}

\begin{itemize}
  \item $f(x)$ should be a polynomial of degree 2 of the form $f(x)=a+bx+cx^2$
  \item The values of x should be equidistant
  \item The number of division of the interval $(x_0,x_0+nh)$ should be
    multiple of 2 like 2, 4, 6, 8, 10, etc.
\end{itemize}
%}}}

\subsection{Simpson's $\frac{3}{8}^{th}$ rule}%{{{
Putting $n=3$ in the general quadrature formula and neglicting all the
difference higher than third order, we get:

\begin{align*}
  I_1 = \int_{x_0}^{x_0+3h}f(x)dx &=h[3y_0+\frac{3}{2}\Delta y_0
                                   +(\frac{27}{3}-\frac{9}{2})
                                     \frac{\Delta^2y_0}{2!}
                                   +(\frac{81}{4}-27+9)
                                     \frac{\Delta^3y_0}{3!}]\\
                                  &=h[3y_0+\frac{3}{2}(y_1-y_0)
                                   +(9-\frac{9}{2})(\frac{y_2-2y_1+y_0}{2})
                                   +(\frac{81}{4}-27+9)\\
                                  & (\frac{y_3-3y_2+3y_1-y_0}{6})]\\
                          \ \\
                                  &= h[3y_0+\frac{9}{2}(y_1-y_0)
                                   +\frac{9}{3}(\frac{y_2-2y_1+y_0}{2})
                                   +\frac{3}{8}(y_3-3_2+3y_1-y_0)]\\
                                  &=\frac{3h}{8}[8y_0+12(y_1-y_0)
                                  +6(y_2-2y_1+y_0)+(y_3-3y_2+3y_1-y_0)]\\
                                  &=\frac{3h}{8}[y_0+3y_1+3y_2+y_3]\\
  \text{Similarly,}\\
                                  & \int_{x_0+3h}^{x_0+6h}f(x)dx
                                  =\frac{3h}{8}[y_3+3y_4+3y_5+y_6]\\
  \dots\\
                                  & \int_{x_0+(n-3)h}^{x_0+nh}f(x)dx 
                                  =\frac{3h}{8}[y_{n-3}+3y_{n-2}
                                  +3y_{n-1}+y_n]\\
  \text{Adding all the integrals,}\\
\end{align*}
\begin{tcolorbox}
\begin{align*}
         \int_{x_0}^{x_0+nh}f(x)dx&=\frac{3h}{8}[(y_0+y_n)
                                   +3(y_1+y_2+y_4+y_5+\dots+y_{n-1})\\
                                  &+2(y_3+y_6+\dots+y_{n-3})]\\
  \text{This is Simpson's $\frac{3}{8}^{rd}$ rule.}
\end{align*}
\end{tcolorbox}


\subsubsection{Conditions for validity of Simpson's $\frac{3}{8}^{rd}$ rule.}
\begin{itemize}
  \item $f(x)$ should be a polynomial of degree 3 of the form 
    $f(x)=a+6x+cx^2+dx^3$
  \item The value of $x$ should be equidistant.
  \item The number of division of the interval $(x_0,x_0+nh)$
    should be multiple of 3 like 3, 6, 9, etc.
\end{itemize}
%}}}
%}}}

\section{Bi-section Method}%{{{
This method of solving a trascendental equation consists in locating the roots of the
equation $f(x)=0$ between two number, say $x_0$ and $x_1$, such that $f(x)$ is
continuous for $x_0\leq x\leq x_1$ and $f(x_0)$ and $f(x_1)$ are the opposite signs
so that $f(x_0)*f(x_1)<0$ IE, the curve crosses the x-axis between $x_0$ and $x_1$
and the desire root of the given equation is approximately,
\begin{tcolorbox}
\begin{align*}
           x_2 &= \frac{x_0+x_1}{2}\\
           x_3 &= \frac{x_0+x_2}{2},\text{provided $f(x_0)*f(x_2)<0$}\\
               &= \frac{x_1+x_2}{2},\text{provided $f(x_1)*f(x_2)<0$}\\
  Further\ x_4 &= \frac{x_0+x_3}{2},\text{provided $f(x_0)*f(x_3)<0$}\\
  \text{and so on\dots}
\end{align*}
\end{tcolorbox}
Thus on each iteration, we either find the root with desired accuracy or we narrow down the range to half of the previous interval.
%}}}

\section{Newton Raphson Method}
The Newton Raphson Method is referred to as one of the most commonly
used techniques for finding the roots of given equations. It can be
efficiently generalised to find solutions to a system of equations.
Moreover, we can show that when we approach the root, the method is
quadratically convergent. In this article, you will learn how to use the
Newton Raphson method to find the roots or solutions of a given equation,
and the geometric interpretation of this method.
\subsection{Newton Raphson Formula}
Let, $x_{0}$ be an approximate value of a root of the
equation $f(x)=0$ and let $(x_{0}+h)$ be the exact value of the
corresponding root where h is very small quantity. Then $f(x_0+h)=0$

Since $x_0+h$ is the root of the equation, $f(x)=0$ by the taylor
series, we can write,
\begin{equation*}
  f(x_0+h)=f(x_0)+hf'(x_0)+\frac{h}{2!}f"(x_0)+\dots = 0
\end{equation*}
where h is very small. Neglecting the 2\textsuperscript{nd} and higher
order terms and taking the first approximation, we get
\begin{align*}
  f(x_0)+hf'(x_0)&=0\\
  \Rightarrow h &= \frac{-f(x_0)}{f'(x_0)}, \text{provided } f'(x_0)\ne 0
\end{align*}

\vskip20pt

\begin{center}
  \centering
  \textbf{\underline{ 1\textsuperscript{st} approximation }}
\end{center}
\begin{equation*}
  x_1=x_0+h=x_0-\frac{f(x_0)}{f'(x_0)}
\end{equation*}

\vskip10pt
\begin{center}
  \textbf{\underline{ 2\textsuperscript{nd} approximation }}
\end{center}
  \begin{equation*}
    x_2=x_1-\frac{f(x_1)}{f'(x_2)}
  \end{equation*}

\end{document}
