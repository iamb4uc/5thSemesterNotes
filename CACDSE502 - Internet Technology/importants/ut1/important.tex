\documentclass[14pt]{article}
% \usepackage{lmodern}
% \renewcommand*\familydefault{\ttdefault} %% Only if the base font of the document is to be typewriter style
% \usepackage[T1]{fontenc}

\usepackage{fullpage}

\usepackage{graphicx}
\graphicspath{ {./images/} }

\usepackage{color}
\usepackage{xcolor}
\usepackage{soul}

\usepackage{listings}

\lstset{
  language=HTML,
  aboveskip=3mm,
  belowskip=3mm,
  showstringspaces=false,
  columns=flexible,
  basicstyle={\normalsize\ttfamily},
  numbers=none,
  breakatwhitespace=true,
  tabsize=3
}

\title{Importants}
\author{Swapnil}

\begin{document}
\maketitle

{\parindent0pt

\section{What is scripting language? What are the different types of scripting languages?***}
A scripting language is a programming language for a special run-time environment that automates the execution of tasks; the tasks can alternative be executed one-by-one by a human operator. 

Scripting languages are:
\begin{enumerate}
  \item Interpreted rather than compiled.
  \item Used for short scripts over full computer programs.
\end{enumerate}
JS, Python, Ruby are all examples of scripting languages.

\section{Difference between client side scripting \& server side\\scripting?***}
\resizebox{\columnwidth}{!}{%
\begin{tabular}{lll}
\multicolumn{1}{c}{\textbf{}} & \multicolumn{1}{c}{\textbf{Client-side-scripting}}                                                                                                      & \multicolumn{1}{c}{\textbf{Server-side-scripting}}                                                                                                 \\ \cline{2-3} 
\multicolumn{1}{l|}{1}        & \multicolumn{1}{l|}{\begin{tabular}[c]{@{}l@{}}Used when the users browser\\ already has the code\end{tabular}}                                         & \multicolumn{1}{l|}{Used to create dynamic pages.}                                                                                                 \\ \cline{2-3} 
\multicolumn{1}{l|}{2}        & \multicolumn{1}{l|}{\begin{tabular}[c]{@{}l@{}}The web browser executes the \\ client side scripting.\end{tabular}}                                     & \multicolumn{1}{l|}{\begin{tabular}[c]{@{}l@{}}The web server executes the server \\ side scripting\end{tabular}}                                  \\ \cline{2-3} 
\multicolumn{1}{l|}{3}        & \multicolumn{1}{l|}{\begin{tabular}[c]{@{}l@{}}Cannot be used to connect to\\ the databases\end{tabular}}                                               & \multicolumn{1}{l|}{\begin{tabular}[c]{@{}l@{}}Used to connect to the databases that\\ reside on the web server\end{tabular}}                      \\ \cline{2-3} 
\multicolumn{1}{l|}{4}        & \multicolumn{1}{l|}{\begin{tabular}[c]{@{}l@{}}Can't access the file system that \\ resides at the web server\end{tabular}}                             & \multicolumn{1}{l|}{\begin{tabular}[c]{@{}l@{}}Can access the file system residing at\\ the web server\end{tabular}}                               \\ \cline{2-3} 
\multicolumn{1}{l|}{5}        & \multicolumn{1}{l|}{\begin{tabular}[c]{@{}l@{}}Response from a client-side script\\ is faster as compared to a server-\\ side script.\end{tabular}}     & \multicolumn{1}{l|}{\begin{tabular}[c]{@{}l@{}}Response from a server-side script is\\ slower as compared to a client-side\\ script.\end{tabular}} \\ \cline{2-3} 
\multicolumn{1}{l|}{6}        & \multicolumn{1}{l|}{\begin{tabular}[c]{@{}l@{}}Example of client-side-script are:\\ VB Script, WML Script, Python, \\ Action Script, etc.\end{tabular}} & \multicolumn{1}{l|}{\begin{tabular}[c]{@{}l@{}}Example of server-side-script are:\\ ASP, JSP, ASP.net, nodejs, PHP, SSJS,\\ etc.\end{tabular}}     \\ \cline{2-3} 
\end{tabular}%
}

\section{Write down the advantages of JS language.}
\large{Advantages of scripting languages}
\begin{enumerate}
  \item \textbf{Easy to learn}:

    The user can learn to code in scripting languages quickly, not much knowledge of web technology is required.

  \item \textbf{Faster Editing}:

    It is highly efficient with the limited number of data structures and variables to use.

  \item \textbf{Interactivity}:

    Helps add visualization interfaces and combinations in web pages.

    Modern web pages demand the use of scripting languages to create enhanced web pages, fascinated visual description which includes background and foreground colors and so on.

  \item \textbf{Functionality}:

    There are different libraries which are part of different scripting languages.
    They help in creating new applications in web browsers and are different from normal programming languages.
\end{enumerate}

\section{What is JS?  Write down the features of JS.}
JS is a dynamic computer programming language. It is a light weight and most commonly scripting language as a part of web pages whose implementations allows client side scripts to interact with user and make dynamic pages. JS is an interpreted programming language with the first
class functions having object-oriented capabilities.

\textbf{Features of JS:}
\begin{enumerate}
  \item \textbf{Speed}

    Since JS is able to run in real-time withing the client-side browser, it is very fast. Unless outside resources are required, JS is unhindered by network called to a backend server.
  \item \textbf{Simplicity}

    Relatively simple to learn and implement.
  \item \textbf{Popularity}

    JS is used everywhere on the web.
  \item \textbf{Interoperability}

    Plays nicely with other languages and can be used in huge variety of applications.
  \item \textbf{Server Load}

    Being client-side reduces the demand on the website server.
\end{enumerate}
\section{What is JS variable? How JS variable are declared?}
JS variables are containers for string data values. JS variables can be declared with the help of `\texttt{var}' keyword.

\textit{example}
\begin{verbatim}
  var x = 5;
  var y = 6;
  var z = x+y;
\end{verbatim}
\section{What is a JS  expression?}
An expression is a valid unit of code that resolves to a value. There are two types of expressions: those that have side effects (such as assigning values) and those that purely evaluate.
\section{What are the different JS operators?}
JS operators are symbols that are used to perform operations on operands.
\textit{example}
\begin{verbatim}
  var sum = 10+20;
\end{verbatim}
Here `+' is the arithmetic operator and `=' is the assignment operator.

Different types of operators used in JS:
\begin{enumerate}
  \item \textbf{Arithmetic Operators}

    \verb|+, -, *, /, ++, --, etc.|

  \item \textbf{Comparison Operators (Relational Operators)}

    \verb|==, <, >, <=, >=, !=, etc.|

  \item \textbf{Bitwise Operators}

    \verb+&, |, ^, ~, <<, >>, etc.+

  \item \textbf{Logical Operators}

    \verb+&&, ||, !, etc.+

  \item \textbf{Assignment Operators}

    \verb|=, +=, -=, *=, etc.|

  \item \textbf{Special Operators}

    \verb|new, delete, void, typeof, etc.|
\end{enumerate}
\section{Write a JS program to find the largest of three number.}
\begin{lstlisting}
<html>
  <head>
    <title>
      JavaScript program to find out the biggest of given three numbers
    </title>
  </head>
  <body>
    <table>
      <tr><td><input type="text" name="a" id="first" placeholder="Enter first number"/></td></tr>
      <tr><td><input type="text" name="b" id="second" placeholder="Enter second number"/></td></tr>
      <tr><td><input type="text" name="c" id="third" placeholder="Enter third number"/></td></tr>
      <tr><td><button onclick = "biggest()">Submit</button> </td> </tr>
    </table>
    
    <div id = "num"></div>
  
    <script type="text/javascript">
    function biggest()
    {
      var a,b,c;
      a = parseInt(document.getElementById ("first").value);
      b = parseInt(document.getElementById ("second").value);
      c = parseInt(document.getElementById ("third").value);
      if(a > b && a > c)
      {
        document.getElementById ("num").innerHTML = a+" is the biggest number";
      }
      else if(b > c && b > a)
      {
        document.getElementById ("num").innerHTML = b+" is the biggest number";
      }
      else
      {
        document.getElementById ("num").innerHTML = c+"is the biggest number";
      }
    }
    </script>
  </body>
</html>
\end{lstlisting}
\section{How to write conditional statement in JS.}
Conditional statements are used to perform different actions based on different conditions.

In JS we have the following conditional statements:
\begin{itemize}
  \item Use \texttt{if} to specify a block of code to be executed, if a specified condition is true
    
    \textit{syntax}
    \begin{lstlisting}
    if (condition) {
    //  block of code to be executed if the condition is true
    }
    \end{lstlisting}
  \item Use \texttt{else} to specify a block of code to be executed, if the same condition is false
    
    \textit{syntax}
    \begin{lstlisting}
    if (condition) {
      //  block of code to be executed if the condition is true
    } else {
      //  block of code to be executed if the condition is false
    }
    \end{lstlisting}
  \item Use \texttt{else if} to specify a new condition to test, if the first condition is false
    
    \textit{syntax}
    \begin{lstlisting}
    if (condition1) {
      //  block of code to be executed if condition1 is true
    } else if (condition2) {
      //  block of code to be executed if the condition1 is false and condition2 is true
    } else {
      //  block of code to be executed if the condition1 is false and condition2 is false
    }
    \end{lstlisting}
  \item Use \texttt{switch} to specify many alternative blocks of code to be executed
    
    \textit{syntax}
    \begin{lstlisting}
    switch(expression) {
      case x:
        // code block
        break;
      case y:
        // code block
        break;
      default:
        // code block
    }
    \end{lstlisting}
\end{itemize}

\section{Program: ODDEVEN***}
\begin{lstlisting}
<html>
  <body>
  Enter First Number <input id="n1"><br>

  <button type = "button" onclick="oddeven()">check</button>
  <p id="demo"></p>
  <script language="javascript" type="text/javascript">
  function oddeven()
  {
    var n;
    n = document.getElementById("n1").value;
    if(n%2==0)
      document.getElementById("demo").innerHTML="the no is even";
    else
      document.getElementById("demo").innerHTML="the no is even";
  }
  </script>
  </body>
</html>
\end{lstlisting}
\section{What is event and how event are handle in JS explain with example.***}
JS events are "things" that happen to HTML elements. When JS is used in HTML pages, JS can "react" on these events.
\ \\

\large \underline{HTML Events}\normalsize

An HTML event can be something the browser does, or something a user does.

Here are some examples of HTML events:
\begin{itemize}
  \item An HTML web page has finished loading
  \item An HTML input field was changed
  \item An HTML button was clicked
\end{itemize}
Often, when events happen, you may want to do something.

JavaScript lets you execute code when events are detected.

HTML allows event handler attributes, with JavaScript code, to be added to HTML elements.

With single quotes:
\begin{lstlisting}
<element event='some JavaScript'>
\end{lstlisting}
With double quotes:
\begin{lstlisting}
<element event="some JavaScript">
\end{lstlisting}
In the following example, an onclick attribute (with code), is added to a \textless button\textgreater element:
\begin{lstlisting}
<button onclick="document.getElementById('demo').innerHTML = Date()">The time is?</button>
\end{lstlisting}
\ \\

\large \underline{Common HTML events}\normalsize

Here is a list of some common HTML events:
\ \\

\resizebox{\columnwidth}{!}{%
\begin{tabular}{|l|l|}
\hline
\multicolumn{1}{|c|}{\textbf{Event}} & \multicolumn{1}{c|}{\textbf{Description}}          \\ \hline
onchange                             & An HTML element has been changed                   \\ \hline
onclick                              & The user clicks an HTML element                    \\ \hline
onmouseover                          & The user moves the mouse over an HTML element      \\ \hline
onmouseout                           & The user moves the mouse away from an HTML element \\ \hline
onkeydown                            & The user pushes a keyboard key                     \\ \hline
onload                               & The browser has finished loading the page          \\ \hline
\end{tabular}%
}

\section{Write a JS program to print “Hello World” processing a button “click me”. Also include a JS function “Say hello”.***}
\begin{lstlisting}
<html>
  <body>
  <input type="button" name=btn value="Click me" onclick="sayHello()">
  <script type="text/javascript">
  function sayHello()
  {
    document.write("<h1>" + "Hello World!" + "</h1>");
  }
  </script>
  </body>
</html>
\end{lstlisting}
\begin{lstlisting}
...
...
...
  <script language="javascript" type="text/javascript">
  function sayHello()
  {
    document.getElementById('p1').innerHTML = "Hello World";
  }
  </script>
  <p id="p1"> </p>
  </body>
</html>
\end{lstlisting}
\section{Write a JS program to add two number, by taking input from HTML form.}
\section{Write a JS program to demonstrate how form validations are perform.}
\section{Write down the structure of a JS?}

}
\end{document}
