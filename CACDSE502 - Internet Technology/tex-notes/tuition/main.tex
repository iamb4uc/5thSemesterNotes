\documentclass[a4paper]{article}

% Packages{{{
  \usepackage[margin=1.2in]{geometry}                         % Margins
    \linespread{1.2}                                          % Increases line spacing
  \usepackage{graphicx}                                       % Image Support
    \graphicspath{ {./images/} }                              % Path to image directory
  \usepackage{color}                                          % Colors for stuff
  \usepackage[dvipsnames]{xcolor}                             % Extra-Colors
  \usepackage{soul}                                           % Provides Highlight feature
  \usepackage{fancyhdr}                                       % Fancy Header and footer
  \usepackage{amsmath}                                        % Math support
  \usepackage{amssymb}                                        % Extended symbols for math
  \usepackage{todonotes}                                        % Extended symbols for math
  \usepackage{listings}                                       % Code-snippet support
    \lstset{                                                % lst-format-config
      frame=tb,
      language=HTML,
      aboveskip=3mm,
      belowskip=3mm,
      showstringspaces=false,
      columns=flexible,
      basicstyle=\ttfamily,
      numbers=none,
      numberstyle=\scriptside\color{Gray},
      keywordstyle=\color{BlueViolet}\slshape,
      commentstyle=\color{ForestGreen},
      stringstyle=\color{Red},
      breaklines=true,
      breakatwhitespace=true,
      tabsize=4
    }
  \usepackage{caption}                                      % Caption support
  \usepackage[utf8]{inputenc}                               % utf encoding
  \usepackage{hyperref}
    \hypersetup{pdfnewwindow=true, colorlinks=false}
%}}}

% Footer section {{{
  % Creates footer{{{
  \pagestyle{fancy}%
  \fancyhf{}%
  \lfoot{Swapnil}%
  \cfoot{iamb4uc.xyz}%
  \rfoot{Page \thepage}%
  \renewcommand{\headrulewidth}{0pt}% Line at the head invisible
  \renewcommand{\footrulewidth}{0.4pt}% Line at the footer visible}}}

  % Redefine the plain page style for chapter pages{{{
  \fancypagestyle{plain}{%
    \fancyhf{}%
    \fancyfoot[L]{Swapnil}%
    \fancyfoot[C]{iamb4uc.xyz}%
    \fancyfoot[R]{Page \thepage}%
    \renewcommand{\headrulewidth}{0pt}% Line at the head invisible
    \renewcommand{\footrulewidth}{0.4pt}% Line at the footer visible
  }%}}}
% }}}

% Title {{{
\title{\Huge{\emph{Internet Technology}} \normalsize{Tuition Notes}}
\author{by \huge{\emph{Swapnil}} \normalsize{written in {\LaTeX}}}
\date{}
% }}}

\begin{document}
%Title page{{{
  \maketitle
  \thispagestyle{empty}
%}}}

% ToC{{{
\setcounter{tocdepth}{3}
\tableofcontents
\pagebreak
%}}}
% -----------------------------------------------------------------------------
\parindent0pt

\part{Unit 2}%{{{
  \subsection{What is markup language?}%{{{
  A computer language that consists of easily understood keywords,
  names, or tags that help format the overall view of a page and 
  the data it contains. Some examples of a markup language are BBC, 
  HTML, SGML, and XML.%}}}

  \subsection{What is HTML? Features of HTML.}%{{{
  The Hyper-Text-Markup-Language or HTML is the standard markup
  language for documents designed to be displayed in a web browser.
  It can be assisted by technologies such as cascading stylesheets
  and scripting languages such as JavaScript. HTML describes the
  structure of a webpage. HTML consist of a series of elements.
  HTML elements tell the browser how to display the content or HTML
  is the standard markup language for documents designed to be
  displayed in a web browser. It can be assisted by technologies such
  as cascading stylesheets and scripting languages such as JavaScript.
  HTML describes the structure of a webpage. HTML consist of a series
  of elements. HTML elements tell the browser how to display the content.%}}}

  \subsection{Difference between HTML and DHTML.}%{{{
  Following are the difference between HTML and DHTML:

  \begin{itemize}
    \item DHTML is a collection of technologies while HTML is a markup
      language.
    \item DHTML creates dynamic webpages where as HTML creates static
      webpages.
    \item DHTML allows including small animations and dynamic menus
      but HTML does not.
    \item DHTML uses events, methods, properties to insulate dynamism in 
      HTML page.
    \item DHTML is basically using JavaScript and stylesheets in an HTML
      page.
    \item DHTML sites will be fast enough upon client-side technologies
      while HTML sites will be slow upon client-side technologies.
    \item DHTML may contained server-side code while HTML cannot have any
      server-side code.
  \end{itemize}%}}}


  \subsection{Different HTML tags}%{{{
  \begin{enumerate}
    \item \texttt{<a>}: It is termed as anchor tag and it creates a hyperlink
      or link.
    \item \texttt{<abbr>}: It defines an abbreviation for a phrase or longer
      word.
    \item \texttt{<acronym>}: It defines acronym for a word.
      (Not supported in HTML5)
    \item \texttt{<address>}: It defines the author's contact information of
      the HTML article
    \item \texttt{<applet>}: It defines an embedded Java applet.
      (Not supported in HTML5)
    \item \texttt{<area>}: It defines the area of an image map.
    \item \texttt{<article>}: It defines the self-contained content.
    \item \texttt{<aside>}: It defines content aside from main content. 
      Mainly represented as sidebar.
    \item \texttt{<audio>}: It is used to embed sound content in HTML document.
    \item \texttt{<b>}: It is used to make a text bold.
    \item \texttt{<base>}: This tag defines the base URL for all relative URL
      within the document.
    \item \texttt{<basefont>}: This tag is used to set default font, size and
      color for all elements of document. (Not supported in HTML5)
    \item \texttt{<bdi>}: This tag is used to provide isolation for that part
      of text which may be formatted in different directions from its 
      surrounding text.
    \item \texttt{<bdo>}: It is used to override the current text direction.
    \item \texttt{<big>}: This tag is used to make font size one level larger
      than its surrounding content. (Not supported in HTML5)
    \item \texttt{<blockquote>}: It is used to define a content which is taken
      from another source.
    \item \texttt{<body>}: It is used to define the body section of an HTML
      document.
    \item \texttt{<br>}: It is used to apply single line break.
    \item \texttt{<button>}: It is used to represent a clickable button
    \item \texttt{<canvas>}: It is used to provide a graphics space within a 
      web document.
    \item \texttt{<caption>}: It is used to define a caption for a table.
    \item \texttt{<center>}: It is used to align the content in center.
      (Not supported in HTML5)
    \item \texttt{<cite>}: It is used to define the title of the work, book,
      website, etc.
    \item \texttt{<code>}: It is used to display a part of programming code in
      an HTML document.
    \item \texttt{<col>}: It defines a column within a table which represent 
      common properties of columns and used with the \texttt{<colgroup>}
      element.
    \item \texttt{<colgroup>}: It is used to define group of columns in a 
      table.
    \item \texttt{<data>}: It is used to link the content with the
      machine-readable translation.
    \item \texttt{<datalist>}: It is used to provide a predefined list for 
      input option.
    \item \texttt{<dd>}: It is used to provide definition/description of a 
      term in description list.
    \item \texttt{<del>}: It defines a text which has been deleted from the 
      document.
    \item \texttt{<details>}: It defines additional details which user can
      either view or hide.
    \item \texttt{<dfn>}: It is used to indicate a term which is defined
      within a sentence/phrase.
    \item \texttt{<dialog>}: It defines a dialog box or other interactive
      components.
    \item \texttt{<dir>}: It is used as container for directory list of files.
      (Not supported in HTML5)
    \item \texttt{<div>}: It defines a division or section within HTML 
      document.
    \item \texttt{<dl>}: It is sued to define a description list.
    \item \texttt{<dt>}: It is used to define a term in description list.
    \item \texttt{<em>}: It is used to emphasis the content applied within 
      this element.
    \item \texttt{<embed>}: It is used as embedded container for external 
      file/application/media, etc.
    \item \texttt{<fieldset>}: It is used to group related elements/labels 
      within a web form.
    \item \texttt{<figcaption>}: It is used to add a caption or explanation 
      for the <figure> element.
    \item \texttt{<figure>}: It is used to define the self-contained content, 
      and s mostly refer as single unit.
    \item \texttt{<font>}: It defines the font, size, color, and face for 
      the content. (Not supported in HTML5)
    \item \texttt{<footer>}: It defines the footer section of a webpage.
    \item \texttt{<form>}: It is used to define an HTML form.
    \item \texttt{<frame>}: It defines a particular area of webpage which 
      can contain another HTML file. (Not supported in HTML5)
    \item \texttt{<frameset>}: It defines group of Frames. 
      (Not supported in HTML5)
    \item \texttt{<h1> to <h6>}: It defines headings for an HTML 
      document from level 1 to level 6.
    \item \texttt{<head>}: It defines the head section of an HTML document.
    \item \texttt{<header>}: It defines the header of a section or webpage.
    \item \texttt{<hr>}: It is used to apply thematic break between 
      paragraph-level elements.
    \item \texttt{<html>}: It represents root of an HTML document.
    \item \texttt{<i>}: It is used to represent a text in some different voice.
    \item \texttt{<iframe>}: It defines an inline frame which can embed other 
      content.
    \item \texttt{<img>}: It is used to insert an image within an HTML 
      document.
    \item \texttt{<input>}: It defines an input field within an HTML form.
    \item \texttt{<ins>}: It represent text that has been inserted within 
      an HTML document.
    \item \texttt{<isindex>}: It is used to display search string for 
      current document. (Not supported in HTML5)
    \item \texttt{<kbd>}: It is used to define keyboard input.
    \item \texttt{<label>}: It defines a text label for the input field of
      form.
    \item \texttt{<legend>}: It defines a caption for content of 
      \texttt{<fieldset>}
    \item \texttt{<li>}: It is used to represent items in list.
    \item \texttt{<link>}: It represents a relationship between current 
      document and an external resource.
    \item \texttt{<main>}: It represents the main content of an HTML document.
    \item \texttt{<map>}: It defines an image map with active areas.
    \item \texttt{<mark>}: It represents a highlighted text.
    \item \texttt{<marquee>}: It is used to insert the scrolling text or an 
      image either horizontally or vertically. (Not supported in HTML5)
    \item \texttt{<menu>}: It is used for creating a menu list of commands.
    \item \texttt{<meta>}: It defines metadata of an HTML document.
    \item \texttt{<meter>}: It defines scalar measurement with known range 
      or fractional value.
    \item \texttt{<nav>}: It represents section of page to represent
      navigation links.
    \item \texttt{<noframes>}: It provides alternate content to represent in 
      browser which does not support the <frame> elements. (Not supported in
      HTML5)
    \item \texttt{<noscript>}: It provides an alternative content if a script 
      type is not supported in browser.
    \item \texttt{<object>}: It is used to embed an object in HTML file.
    \item \texttt{<ol>}: It defines an ordered list of items.
    \item \texttt{<optgroup>}: It is used to group the options of a drop-down 
      list.
    \item \texttt{<option>}: It is used to define options or items in a 
      drop-down list.
    \item \texttt{<output>}: It is used as container element which can show 
      result of a calculation.
    \item \texttt{<p>}: It represents a paragraph in an HTML document.
    \item \texttt{<param>}: It defines parameter for an <object> element
    \item \texttt{<picture>}: It defines more than one source element and 
      one image element.
    \item \texttt{<pre>}: It defines preformatted text in an HTML document.
    \item \texttt{<progress>}: It defines the progress of a task within HTML 
      document.
    \item \texttt{<q>}: It defines short inline quotation.
    \item \texttt{<rp>}: It defines an alternative content if browser does 
      not supports ruby annotations.
    \item \texttt{<rt>}: It defines explanations and pronunciations in ruby 
      annotations.
    \item \texttt{<ruby>}: It is used to represent ruby annotations.
    \item \texttt{<s>}: It render text which is no longer correct or relevant.
    \item \texttt{<samp>}: It is used to represent sample output of a computer
      program.
    \item \texttt{<script>}: It is used to declare the JavaScript within HTML
      document.
    \item \texttt{<section>}: It defines a generic section for a document.
    \item \texttt{<select>}: It represents a control which provides a menu
      of options.
    \item \texttt{<small>}: It is used to make text font one size smaller
      than document's base font size.
    \item \texttt{<source>}: It defines multiple media recourses for different
      media element such as \texttt{<picture>}, \texttt{<video>}, and
      \texttt{<audio>} element.
    \item \texttt{<span>}: It is used for styling and grouping inline.
    \item \texttt{<strike>}: It is used to render strike through the text.
      (Not supported in HTML5)
    \item \texttt{<strong>}: It is used to define important text.
    \item \texttt{<style>}: It is used to contain style information for
      an HTML document.
    \item \texttt{<sub>}: It defines a text which displays as a 
      subscript text.
    \item \texttt{<summary>}: It defines summary which can be used
      with <details> tag.
    \item \texttt{<sup>}: It defines a text which represent as
      superscript text.
    \item \texttt{<svg>}: It is used as container of SVG
      (Scalable Vector Graphics).
    \item \texttt{<table>}: It is used to present data in tabular form
      or to create a table within HTML document.
    \item \texttt{<tbody>}: It represents the body content of an HTML
      table and used along with <thead> and <tfoot>.
    \item \texttt{<td>}: It is used to define cells of an HTML table
      which contains table data
    \item \texttt{<template>}: It is used to contain the client side
      content which will not display at time of page load and may render
      later using JavaScript.
    \item \texttt{<textarea>}: It is used to define multiple line input,
      such as comment, feedback, and review, etc.
    \item \texttt{<tfoot>}: It defines the footer content of an HTML table.
    \item \texttt{<th>}: It defines the head cell of an HTML table.
    \item \texttt{<thead>}: It defines the header of an HTML table.
      It is used along with \texttt{<tbody>} and \texttt{<tfoot>} tags.
    \item \texttt{<time>}: It is used to define data/time within an 
      HTML document.
    \item \texttt{<title>}: It defines the title or name of an HTML document.
    \item \texttt{<tr>}: It defines the row cells in an HTML table
    \item \texttt{<track>}: It is used to define text tracks for
      \texttt{<audio>} and \texttt{<video>} elements.
    \item \texttt{<tt>}: It is used to define teletype text.
      (Not supported in HTML5)

    \item \texttt{<u>}: It is used to render enclosed text with an underline.
    \item \texttt{<ul>}: It defines unordered list of items.

    \item \texttt{<var>}: It defines variable name used in mathematical
      or programming context.
    \item \texttt{<video>}: It is used to embed a video content with an 
      HTML document

    \item \texttt{<wbr>}: It defines a position within text where breakline
      is possible.
  \end{enumerate}%}}}

  \subsection{What is a webpages?}%{{{
  A webpage is a document, commonly written in HTML, that is viewed
  in an internet browser. A webpage can be accessed by entering a URL
  address into a browser's address bar. A webpage may contain text,
  graphics and hyperlinks to other webpages and files.%}}}

  \subsection{What is a website?}%{{{
  A website is a collection of web pages and related content that 
  is identified by a common domains name and published on at least
  one webserver.%}}}

  \subsection{Difference between webpages, websites and webservers}%{{{
  \textbf{Webpages}: A document which can be displayed in a web browser
  such as Mozzila Firefox, Google Chrome, Opera, Microsoft Edge, Apple 
  Safari. These are also often called just `pages'.

  \textbf{Website}: A collection of webpages which are grouped together
  and usually connected together in various ways. Often called a `website'
  or a `site'.

  \textbf{Webserver}: A computer that hosts a website on the internet.%}}}

  \subsection{What are HTML tags}%{{{
  HTML tags are the hiddent keyword within a webpages that define how
  our web browser must format and display the content. Most tags must
  have two parts:

  \begin{enumerate}
    \item An openning
    \item A closing part
  \end{enumerate}

  \texttt{<html>} is the openning tag and \texttt{</html>} is a closing
  tag. An opening tag begins a section of page content, and a closing 
  tag ends it.%}}}

  \subsection{DHTML}%{{{
  It is a collection of technologies used together to create interactive
  and animated website by using a combination of a static markup language
  such as HTML, a client-side scripting language such as Javascript, 
  VBScript etc., a presentation defination language such as CSS (\emph
  {cascading style sheets}) and the document object model (\emph
  {DOM})

  The application of DHTML was introduced by microsoft with the release
  of internet explorer 4.

  DHTML allows scripting languages to change variables in a web page's 
  definition language, which in turn affects the look and function of 
  otherwise "static" HTML page content after the page has been fully 
  loaded and during the viewing process. Thus the dynamic characteristic 
  of DHTML is the way it functions while a page is viewed, not in its 
  ability to generate a unique page with each page load.
    \subsubsection{Features}
    DHTML allows authors to add effects to their pages that are otherwise
    difficult to achieve by changing the document object model (\emph{DOM})
    and page style. The combination of HTML, CSS and Javascript offers
    ways to:
    \begin{itemize}
      \item Animate text and images in their documents.
      \item Embedded a ticker or other dynamic display that automatically
        refreshes its content with the latest news, stocks, quotes or 
        other data.
      \item Use a form to capture user input and then process verify
        and respond to that data without having to send data back to the 
        server.
      \item It is used for real time positioning, dynamic fonts.
      \item Include rollover buttons or drop-down menu's.
      \item It is used for data binding.
    \end{itemize}%}}}

  \subsection{Difference between HTML and XML}%{{{
  \begin{center}
    \begin{tabular}{l|l|l|}
    \cline{2-3}
                   &
      \textbf{XML} &
      \textbf{HTML} \\ 
      \hline

       
    \multicolumn{1}{|l|}{\textit{1}} &
      \begin{tabular}[c]{@{}l@{}}
        It stores and transports\\ 
        data. 
      \end{tabular}&

      It displays data. \\ 
      \hline


    \multicolumn{1}{|l|}{\textit{2}} &
      It uses user-defined tags. &

      It uses predefined tags. \\ 
      \hline


    \multicolumn{1}{|l|}{\textit{3}} &
      It contains structural data. &

      \begin{tabular}[c]{@{}l@{}}
      It does not contains any \\ 
      structural data.
      \end{tabular} \\ 
      \hline


    \multicolumn{1}{|l|}{\textit{4}} &
      \begin{tabular}[c]{@{}l@{}}
      It can distinguish upper case\\ 
      and lower case letter.
      \end{tabular} &

      \begin{tabular}[c]{@{}l@{}}
      It cannot distinguish upper\\ 
      case and lower case letter.
      \end{tabular} \\ 
      \hline


    \multicolumn{1}{|l|}{\textit{5}} &
      \begin{tabular}[c]{@{}l@{}}
      It maintains spacing, tabs, \\ 
      new lines, and any other white\\ 
      space formatting.
      \end{tabular} &

      \begin{tabular}[c]{@{}l@{}}
      It does not maintains white\\ 
      space.
      \end{tabular} \\ 
      \hline

    \end{tabular}%
  \end{center}
    %}}}

  \subsection{Stylesheets}%{{{
  A web style sheet is a form of separation of presentation and content
  for web design in which the markup of a webpage contains the page's 
  semantic content and structure but does not defines its visual layout.
  It is a collection of style rule that tells a browser how the various 
  styles are to be applied to the HTML tags to present the document.

  There are three types of stylesheets which are given below:
  \begin{itemize}
    \item \textbf{Inline Stylesheet}:

      Inline stylesheets contains the 
      stylesheet property in the body section attached with element
      is known as inline stylesheet. This kind of stye is specified
      within an HTML tag using the style attribute.

    \item \textbf{Internal or Embedded Stylesheet}:

      This can be used when a single HTML document must be styled 
      uniquely. The stylesheet rule set should be within the HTML file in 
      the head section i.e. the stylesheet

      
  \end{itemize}

    \subsubsection{CSS}
    Cascading style sheets, fondly referred to as CSS, is a simply 
    designed language, intended to simplify the process of making 
    web pages presentable. CSS allows you to apply styles to webpages.
    More importantly, CSS enables you to do this independent of the 
    HTML that makes up each webpage.
      \subsection{Features}
      \begin{enumerate}
        \item CSS saves time
        \item Easy Maintenance
        \item Search Engines
        \item Superior styles to HTML
        \item Offline Browsing.
      \end{enumerate}%}}}

  \subsection{How to format text using CSS?}%{{{
  CSS allows us to format text to create visually appealing contents
  The following properties are used to style text using CSS.
  \textit{For example}:
  \begin{itemize}
    \item \texttt{text-color}
    \item \texttt{text-alignment}
    \item \texttt{letterspacing}
    \item \texttt{lineheight}
    \item \texttt{text-alignment}
    \item \texttt{text-decoration}
    \item \texttt{text-indent}
    \item \texttt{text-shadow}
    \item \texttt{text-transform}
    \item \texttt{wordspacing}
  \end{itemize}%}}}

  \subsection{Design an HTML form which contains text-input controls,%{{{
  checkbox controls, radio button control. select-box control, file-upload
  box, button controls.}
  \begin{lstlisting}
  \end{lstlisting}%}}}

  \subsection{HTML forms}
  An HTML form is a section of a document which contains controls such as 
  text fields, password fields, checkboxes, radio buttons, submit button, 
  menus etc.

  An HTML form facilitates the user to enter data that is to be sent to the 
  server for processing such as name, email address, password, phone number,
  etc.

    \subsubsection{Why we use HTML form?}
    HTML forms are required if you want to collect some data from of the site 
    visitor. \textit{Example}: If a user want to purchase some items on 
    internet, he/she must fill the form such as shipping address and
    credit/debit card details so that item can be sent to the given address.

    \subsubsection{Syntax of HTML form}
    HTML form syntax is as follows:
    \begin{lstlisting}
    <form>
      form elements
    </form>
    \end{lstlisting}



%}}}

\pagebreak

\part{Unit 3}%{{{
  \subsection{Scripting Language} %{{{
  A scripting language is a programming language for a special run-time
  environment that automates the execution of tasks; the tasks can 
  alternative be executed one-by-one by a human operator.

  Scripting languages are:
  \begin{enumerate}
    \item Interpreted rather than compiled.
    \item Used for short scripts over full computer programs.
  \end{enumerate}%}}}

  \subsection{Different types of scripting languages}%{{{
  JS, Python, Ruby are all examples of scripting languages.%}}}

  \subsection{Difference between client-side scripting \& %{{{
  server-side scripting?}
  \begin{center}
  
  \begin{tabular}{lll}

  \multicolumn{1}{c}{\textbf{}} & 

    \multicolumn{1}{c}
    {\textbf{Client-side-scripting}} &

    \multicolumn{1}{c}
    {\textbf{Server-side-scripting}}\\
    \cline{2-3}


  \multicolumn{1}{l|}{1} & 

    \multicolumn{1}{l|}
    {\begin{tabular}[c]{@{}l@{}}
    Used when the users browser \\
    already has the code
    \end{tabular}} & 

    \multicolumn{1}{l|}
    {Used to create dynamic pages.} \\ 
    \cline{2-3} 


  \multicolumn{1}{l|}{2} & 

    \multicolumn{1}{l|}
    {\begin{tabular}[c]{@{}l@{}}
    The web browser executes the \\
    client side scripting.
    \end{tabular}} & 

    \multicolumn{1}{l|}
    {\begin{tabular}[c]{@{}l@{}}
    The web server executes the server \\ 
    side scripting
    \end{tabular}}\\ 
    \cline{2-3} 


  \multicolumn{1}{l|}{3} &

    \multicolumn{1}{l|}
    {\begin{tabular}[c]{@{}l@{}}
    Cannot be used to connect to\\ 
    the databases
    \end{tabular}} &

    \multicolumn{1}{l|}
    {\begin{tabular}[c]{@{}l@{}}
    Used to connect to the databases that \\ 
    reside on the web server
    \end{tabular}} \\
    \cline{2-3} 


  \multicolumn{1}{l|}{4} &

    \multicolumn{1}{l|}
    {\begin{tabular}[c]{@{}l@{}}
    Can't access the file system that \\ 
    resides at the web server
    \end{tabular}} & 

    \multicolumn{1}{l|}
    {\begin{tabular}[c]{@{}l@{}}
    Can access the file system residing at\\ 
    the web server
    \end{tabular}}\\
    \cline{2-3} 


  \multicolumn{1}{l|}{5} &

    \multicolumn{1}{l|}
    {\begin{tabular}[c]{@{}l@{}}
    Response from a client-side script\\
    is faster as compared to a server-\\
    side script.
    \end{tabular}} & 

    \multicolumn{1}{l|}
    {\begin{tabular}[c]{@{}l@{}}
    Response from a server-side script is\\ 
    slower as compared to a client-side\\ 
    script.
    \end{tabular}} \\
    \cline{2-3} 


  \multicolumn{1}{l|}{6}&

    \multicolumn{1}{l|}
    {\begin{tabular}[c]{@{}l@{}}
    Example of client-side-script are:\\
    VB Script, WML Script, Python, \\
    Action Script, etc.
    \end{tabular}} &

    \multicolumn{1}{l|}
    {\begin{tabular}[c]{@{}l@{}}
    Example of server-side-script are:\\
    ASP, JSP, ASP.net, nodejs, PHP, SSJS,\\
    etc.\end{tabular}} \\
    \cline{2-3} 

  \end{tabular}
  \end{center}
  %}}}

  \subsection{Advantages of JS}%{{{
    \subsubsection{Easy to learn}
    The user can learn to code in scripting languages quickly, not 
    much knowledge of web technology is required.

    \subsubsection{Faster Editing}
    It is highly efficient with the limited number of data structures 
    and variables to use.

    \subsubsection{Interactivity}
    Helps add visualization interfaces and combinations in web pages.

    Modern web pages demand the use of scripting languages to create 
    enhanced web pages, fascinated visual description which includes 
    background and foreground colors and so on.

    \subsubsection{Functionality}
    There are different libraries which are part of different 
    scripting languages.
    They help in creating new applications in web browsers and are 
    different from normal programming languages.%}}}

  \subsection{JavaScript}%{{{
  JS is a dynamic computer programming language. It is a light
  weight and most commonly scripting language as a part of web pages
  whose implementations allows client side scripts to interact with
  user and make dynamic pages. JS is an interpreted programming
  language with the first class functions having object-oriented
  capabilities.
    \subsubsection{Features}%{{{
      \begin{enumerate}
        \item \textbf{Speed}: Since JS is able to run in real-time 
          withing the client-side browser, it is very fast. Unless
          outside resources are required, JS is unhindered by 
          network called to a backend server.
        \item \textbf{Simplicity}: Relatively simple to learn 
          and implement.
        \item \textbf{Popularity}: JS is used everywhere on the 
          web.
        \item \textbf{Interoperability}: Plays nicely with other 
          languages and can be used in huge variety of applications.
        \item \textbf{Server Load}: Being client-side reduces the
          demand on the website server.
      \end{enumerate}%}}}

    \subsubsection{JavaScript variable}%{{{
    JS variables are containers for string data values. JS variables
    can be declared with the help of `\texttt{var}' keyword.

    \emph{Example}
    \begin{lstlisting}
      var x = 5;
      var y = 6;
      var z = x+y;
    \end{lstlisting}%}}}

    \subsubsection{JavaScript expression}%{{{
    An expression is a valid unit of code that resolves to a value.
    There are two types of expressions: those that have side effects
    (such as assigning values) and those that purely evaluate.%}}}

    \subsubsection{Different JavaScript operators?}%{{{
    JavaScript operators are symbols that are used to perform 
    operations on operands.

    \textit{Example}
    \begin{lstlisting}
      var sum = 10 + 20;
    \end{lstlisting}
    Here `\texttt{+}' is the arithmetic operator and `\texttt{=}' is the assignment operator.

    Different types of operators used in JavaScript:
    \begin{enumerate}
      \item \textbf{Arithmetic Operators}:
        \verb|+, -, *, /, ++, --, etc.|
      \item \textbf{Comparison Operators (Relational Operators)}:
        \verb|==, <, >, <=, >=, !=, etc.|
    
      \item \textbf{Bitwise Operators}:
        \verb+&, |, ^, ~, <<, >>, etc.+
    
      \item \textbf{Logical Operators}:
        \verb+&&, ||, !, etc.+
    
      \item \textbf{Assignment Operators}:
        \verb|=, +=, -=, *=, etc.|
    
      \item \textbf{Special Operators}:
        \verb|new, delete, void, typeof, etc.|
    \end{enumerate}%}}}
%}}}

  \subsection{Write a JS program to find the largest of three number.}%{{{
  \begin{lstlisting}
  <html>
    <head>
      <title>
        JavaScript program to find out the biggest of given three numbers
      </title>
    </head>
    <body>
      <table>
        <tr><td><input type="text" name="a" id="first" placeholder="Enter first number"/></td></tr>
        <tr><td><input type="text" name="b" id="second" placeholder="Enter second number"/></td></tr>
        <tr><td><input type="text" name="c" id="third" placeholder="Enter third number"/></td></tr>
        <tr><td><button onclick = "biggest()">Submit</button> </td> </tr>
      </table>
      
      <div id = "num"></div>
    
      <script type="text/javascript">
      function biggest()
      {
        var a,b,c;
        a = parseInt(document.getElementById ("first").value);
        b = parseInt(document.getElementById ("second").value);
        c = parseInt(document.getElementById ("third").value);
        if(a > b && a > c)
        {
          document.getElementById ("num").innerHTML = a+" is the biggest number";
        }
        else if(b > c && b > a)
        {
          document.getElementById ("num").innerHTML = b+" is the biggest number";
        }
        else
        {
          document.getElementById ("num").innerHTML = c+"is the biggest number";
        }
      }
      </script>
    </body>
  </html>
  \end{lstlisting}%}}}

  \subsection{Conditional statement in JavaScript}%{{{
  Conditional statements are used to perform different actions
  based on different conditions.

  In JS we have the following conditional statements:
  \begin{itemize}
    \item Use \texttt{if} to specify a block of code to be 
      executed, if a specified condition is true

      \textit{syntax}
      \begin{lstlisting}
      if (condition) {
      //  block of code to be executed if the condition is true
      }
      \end{lstlisting}
    \item Use \texttt{else} to specify a block of code to be 
      executed, if the same condition is false

      \textit{syntax}
      \begin{lstlisting}
      if (condition) {
        //  block of code to be executed if the condition is true
      } else {
        //  block of code to be executed if the condition is false
      }
      \end{lstlisting}
    \item Use \texttt{else if} to specify a new condition to test, 
      if the first condition is false

      \textit{syntax}
      \begin{lstlisting}
      if (condition1) {
        //  block of code to be executed if condition1 is true
      } else if (condition2) {
        //  block of code to be executed if the condition1 is false and condition2 is true
      } else {
        //  block of code to be executed if the condition1 is false and condition2 is false
      }
      \end{lstlisting}
    \item Use \texttt{switch} to specify many alternative blocks 
      of code to be executed
      
      \textit{syntax}
      \begin{lstlisting}
      switch(expression) {
        case x:
          // code block
          break;
        case y:
          // code block
          break;
        default:
          // code block
      }
      \end{lstlisting}
  \end{itemize}%}}}

  \subsection{Program: ODDEVEN}%{{{
  \begin{lstlisting}
  <html>
    <body>
    Enter First Number <input id="n1"><br>
  
    <button type = "button" onclick="oddeven()">check</button>
    <p id="demo"></p>
    <script language="javascript" type="text/javascript">
    function oddeven()
    {
      var n;
      n = document.getElementById("n1").value;
      if(n%2==0)
        document.getElementById("demo").innerHTML="the no is even";
      else
        document.getElementById("demo").innerHTML="the no is even";
    }
    </script>
    </body>
  </html>
  \end{lstlisting}%}}}

  \subsection{What are events? How events are handled in JavaScript %{{{
  explain with example.}
  JS events are "things" that happen to HTML elements. When JS is 
  used in HTML pages, JS can "react" on these events.

  \begin{center}
    \textbf{HTML Events}
  \end{center}
  An HTML event can be something the browser does, or something 
  a user does.

  Here are some examples of HTML events:
  \begin{itemize}
    \item An HTML web page has finished loading
    \item An HTML input field was changed
    \item An HTML button was clicked
  \end{itemize}
  Often, when events happen, you may want to do something.
  JavaScript lets you execute code when events are detected.
  HTML allows event handler attributes, with JavaScript code,
  to be added to HTML elements.

  With single quotes:

  \begin{lstlisting}
  <element event='some JavaScript'>
  \end{lstlisting}

  With double quotes:

  \begin{lstlisting}
  <element event="some JavaScript">
  \end{lstlisting}

  In the following example, an onclick attribute (with code),
  is added to a \textless button\textgreater element:

  \begin{lstlisting}
  <button onclick="document.getElementById('demo').innerHTML = Date()">The time is?</button>
  \end{lstlisting}

  \begin{center}
    \textbf{Common HTML Events}
  \end{center}

  \begin{center}
  Here is a list of some common HTML events:
  \end{center}

  \begin{center}
  \begin{tabular}{|l|l|}
  \hline

  \multicolumn{1}{|c|}
    {\textbf{Event}} & 
    \multicolumn{1}{c|}
    {\textbf{Description}}\\ 
    \hline

  onchange & 
    An HTML element has been changed\\ 
    \hline

  onclick &
    The user clicks an HTML element\\
    \hline

  onmouseover &
    The user moves the mouse over an HTML element\\
    \hline

  onmouseout &
    The user moves the mouse away from an HTML element \\
    \hline

  onkeydown &
    The user pushes a keyboard key\\
    \hline

  onload &
    The browser has finished loading the page\\
    \hline

  \end{tabular}%
  \end{center}%}}}

  \subsection{Write a JavaScriptprogram to print “Hello World” processing%{{{
  a button “click me”. Also include a JS function “Say hello”}
  \begin{lstlisting}
  <html>
    <body>
    <input type="button" name=btn value="Click me" onclick="sayHello()">
    <script type="text/javascript">
    function sayHello()
    {
      document.write("<h1>" + "Hello World!" + "</h1>");
    }
    </script>
    </body>
  </html>
  \end{lstlisting}
  \begin{lstlisting}
  ...
  ...
  ...
    <script language="javascript" type="text/javascript">
    function sayHello()
    {
      document.getElementById('p1').innerHTML = "Hello World";
    }
    </script>
    <p id="p1"> </p>
    </body>
  </html>
  \end{lstlisting}%}}}

%}}}

\end{document}
