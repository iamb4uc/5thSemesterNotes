\documentclass{article}
\usepackage{fullpage}

\usepackage{graphicx}
\graphicspath{ {./images/} }

\usepackage{color}
\usepackage{soul}

\usepackage{listings}

\lstset{
  language=Java,
  aboveskip=3mm,
  belowskip=3mm,
  showstringspaces=false,
  basicstyle={\small\ttfamily},
  breaklines=true,
  breakatwhitespace=true,
  tabsize=3
}

\title{Java Importants}
\author{Swapnil}


\begin{document}
\maketitle
{\parindent0pt

\section{Why java ? Explain the features of java.}
Java is one of the most widely used object-oriented programming language and software platform that was built to run on billions of devices. Java has a major advantage with its portability. Once you have written the code for the Java program on a device, it is very easy to move the code to another
device without worrying about compatibility issues. It rules and syntax is inspired from C and C++ and its primary goal is to be able to `write one, run anywhere'.
\ \\

Following are the features of Java:
\begin{enumerate}
  \item \textbf{Platform Independent}

    Java is not compiled for platform specific machine, rather for platform-independent byte-code. This byte-code is distributed over the web and interpreted by the Java Virtual Machine (JVM) on whichever platform it is being run on.

  \item \textbf{Object-oriented}

    Java can be easily extended since it is based on the Object model.

  \item \textbf{Multithreaded}

    With Java's multithreaded feature it is possible to write programs that can perform many tasks simultaneously. This design feature allows the developers to construct interactive applications that can run smoothly.

  \item \textbf{Robust}

    Java makes an effort to eliminate error-prone situations by emphasizing mainly on compile time error checking and runtime checking.

  \item \textbf{Distributed}

    Java is designed for the distributed environment of the internet.
\end{enumerate}

\section{What is byte code?}
Byte-codes are nothing but a platform independent executable file that provides machine instructions for a Java Processor chip called Java Virtual Machine.

\section{What is JVM?}
Programs written in Java are compiled into Java byte-code which is then interpreted by a special Java interpreter for a specific platform. Actually this Java interpreter is known as the Java Virtual Machine (JVM). This machine is called Byte-code. Actually the Java interpreter running on any
machine appears and behaves like a virtual processor chip, that is why -- Java Virtual Machine.
\section{What is JDK?}
The Java development kit comes with the collection of tools that are used for developing and running Java Programs. They include:

\begin{enumerate}
  \item Appletviewer
  \item Java C (Java Compiler)
  \item Java (Java interpreter)
  \item Java D (Java Disassembler)
  \item Java h (for C header files)
  \item Java doc (for creating HTML documents)
  \item Jdb (Java debugger)
\end{enumerate}

\section{How arrays are declared and initialized in Java (1D, 2D). Give example.}
\begin{lstlisting}
/* 1D array */
import java.util.*;

public class oneDarray {
  public static void main(String[] args) {
    int a[], i, n;
    Scanner in = new Scanner(System.in);
    try {
      System.out.println("Enter how many number");
      n=in.nextInt();
      System.out.println("Enter array elements ");
      a=new int[n];
      for (i=0; i<n; i++)
          a[i]=in.nextInt();
      System.out.println("Tje array elements are: ");
      for (i=0; i<n; i++)
          System.out.print(a[i]+" ");
      } catch(Exception e) {
    }
  }
}


/* 2D array */
import java.util.*;

public class twoDarray {
    public static void main(String[] args) {
        int a[][], r, c, i, j;
        Scanner in = new Scanner(System.in);
        try
        {
          System.out.println("Enter row and column for 2D array");
          r=in.nextInt();
          c=in.nextInt();
          System.out.println("Enter 2D elements");
          a=new int[r][c];
          for(i=0; i<r; i++)
            for(j=0; j<c; j++)
              a[i][j]=in.nextInt();
          System.out.println("2D elements are: ");
          for(i=0; i<r; i++)
          {
            for(j=0; j<c; j++)
              System.out.print(a[i][j]+" ");
            System.out.println();
          }
          System.out.println("The row lenght of 2D array is"+a.length);
          System.out.println("The row lenght of 2D array is"+a[0].length);
        } catch(Exception e) {}
  }
}
\end{lstlisting}

\section{Find the sum of any number of integers entered as command line arguments.}
\begin{lstlisting}
package csjava;

public class cmdArg {
  public static void main(String[] args) {
    int cnt, i=0, n, s=0;
    cnt=args.length;
    while(i<cnt) {
      n=Integer.parseInt(args[i]);
      s=s+n;
      i++;

    }
    System.out.println("The sum of integer is "+s);
  }
}
\end{lstlisting}

}
\end{document}
