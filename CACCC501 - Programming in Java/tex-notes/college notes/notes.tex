\documentclass[11pt, a4paper, oneside]{book}

% {{{ Packages
        \usepackage[margin=1in]{geometry}                           % Margins
          \linespread{1.2}                                          % Increases line spacing
        \usepackage{graphicx}                                       % Image Support
          \graphicspath{ {./images/} }                              % Path to image directory
        \usepackage{color}                                          % Colors for stuff
        \usepackage{soul}                                           % Provides Highlight feature
        \usepackage{fancyhdr}                                       % Fancy Header and footer
        \usepackage{amsmath}                                        % Math support
        \usepackage{amssymb}                                        % Extended symbols for math
        \usepackage{todonotes}                                        % Extended symbols for math
        \usepackage{caption}
        % \usepackage{etoolbox}                                       % IDK what it does. I just coppied it from stackexchage
        %   \AtBeginEnvironment{align}{\setcounter{equation}{0}}      % resets align counter for each instance
        \usepackage{listings}                                       % Code-snippet support
          \definecolor{dkgreen}{rgb}{0,0.6,0}                     % lst-configs
          \definecolor{gray}{rgb}{0.5,0.5,0.5}
          \definecolor{mauve}{rgb}{0.58,0,0.82}

          \lstset{                                                % lst-format-config
            frame=tb,
            language=Java,
            aboveskip=3mm,
            belowskip=3mm,
            showstringspaces=false,
            columns=flexible,
            basicstyle={\ttfamily},
            numbers=left,
            numberstyle=\small\color{gray},
            keywordstyle=\color{blue},
            commentstyle=\color{dkgreen},
            stringstyle=\color{mauve},
            breaklines=true,
            breakatwhitespace=true,
            tabsize=3
          }
        \usepackage{caption}                                      % Caption support
        \usepackage[utf8]{inputenc}                               % utf encoding
        \usepackage{hyperref}
          \hypersetup{pdfnewwindow=true, colorlinks=false}
% }}}

% {{{ Footer section
        % Creates footer
        \pagestyle{fancy}%
        \fancyhf{}%
        \lfoot{Swapnil}%
        \cfoot{iamb4uc.xyz}%
        \rfoot{Page \thepage}%
        \renewcommand{\headrulewidth}{0pt}% Line at the head invisible
        \renewcommand{\footrulewidth}{0.4pt}% Line at the footer visible

        % Redefine the plain page style for chapter pages
        \fancypagestyle{plain}{%
          \fancyhf{}%
          \fancyfoot[L]{Swapnil}%
          \fancyfoot[C]{iamb4uc.xyz}%
          \fancyfoot[R]{Page \thepage}%
          \renewcommand{\headrulewidth}{0pt}% Line at the head invisible
          \renewcommand{\footrulewidth}{0.4pt}% Line at the footer visible
        }
% }}}

% Title{{{

        \title{\Huge{\texttt{Programming in Java}} \large{college notes}}
        \author{\huge{by \emph{Swapnil}}\\ \\ \Large{written in {\LaTeX}}}
        \date{\today}
%}}}


\begin{document}%{{{

%{{{ Title page
      \maketitle
      \thispagestyle{empty}
      % \newpage
% }}}

%{{{ ToC
      \setcounter{tocdepth}{3}
      \tableofcontents
% }}}



%%%%%%%%%%%%%%%%%%%%%%%%%%%%%%%%%%%%%%%%%%%%%%%%%%%%%%%%%%%%%%%%%%%%%%%%%%%%%%%%%%%%%%%%%%%
%%%%%%%%%%%%%%%%%%%%%%%%%%%%%%%%%%%%%% Document Body %%%%%%%%%%%%%%%%%%%%%%%%%%%%%%%%%%%%%%
%%%%%%%%%%%%%%%%%%%%%%%%%%%%%%%%%%%%%%%%%%%%%%%%%%%%%%%%%%%%%%%%%%%%%%%%%%%%%%%%%%%%%%%%%%%
\section{How to implement package in Java}
\begin{itemize}
  \item First we create a package named empPack (\textit{say})
  \item Create class file ie create a Java file test, test1 (\textit{say})
    \begin{lstlisting}%{{{

      package empPack;

      class test
      {
        public void disp_test()
        {
          System.out.println("Hello display test ");
        }
      }
      public class test1 extends test
      {
        public void disp_test1()
        {
          System.out.println("Hello display test 1");
        }
      }
    \end{lstlisting}%}}}

  \item Now again create a Java file having main method where we will
    import the above package
    \begin{lstlisting}%{{{

      import empPack.*;
      public class packtest22 {
        public static void main(String[] args) {
          test1 t=new test1();
          t.disp_test();
          t.disp_test1();
        }
      }
    \end{lstlisting}%}}}

\end{itemize}

\section{Practical Assignment}
\begin{itemize}
\item Write a Java program to show how packages are create and
  imported to another Java program.
\item Write a Java program to implement the concept of multiple
  inheritence through interface

    \begin{lstlisting}%{{{

    interface I {
      final static int x=13;
      void disp_2();
    }

    \end{lstlisting}%}}}

\item Implement multilevel inheritence.
\end{itemize}

\chapter{Unit 3}
\section{Exception Handling}
\begin{itemize}
  \item Mechanism to handle the runtime errors such as \texttt{ClassNotFoundException}, \texttt{IOException}, \texttt{SQLException}, \texttt{RemoteException}.
  \item An exception is an event that disrupts the normal flow of the program.
  \item It is an object which his thrown at runtime.
\end{itemize}

\subsection{Types of exceptions}
\begin{itemize}
  \item \textit{Checked}

    Checked Exceptions are checked at compile time. Example \texttt{IOException}, \texttt{SQLException},

  \item \textit{Unchecked Exceptions}

    Are not checked at compile time but checked at runtime.
\end{itemize}

\subsection{Keywords used in exception handling}
\begin{itemize}
  \item \texttt{try}
  \item \texttt{catch}
  \item \texttt{finally}
  \item \texttt{throw}
  \item \texttt{throws}

\end{itemize}

\subsection{Multithreaded programming}
\begin{itemize}
  \item (\textit{extend}) Thread Class
  \item (\textit{implement}) Roundable
\end{itemize}

\newpage
\centering{\emph{\small{Thread Example}}}
\begin{lstlisting}
package csjava;

class tA extends Thread {
  public void run() {
    for{i=1;i<=5;i++}
      System.out.println("i = "+i)
  }
}

class tB extends Thread {
  public void run() {
    for{j=1;j<=5;j++}
      System.out.println("i = "+i)
  }
}

class tC extends Thread {
  public void run() {
    for{k=1;k<=5;k++}
      System.out.println("i = "+i)
  }
}

public class threadExample {
  public static void main(String[] args) {
    tA A = new tA();
    tB B = new tB();
    tC C = new tC();

    A.start();
    B.start();
    C.start();
  }
}
\end{lstlisting}


\centering{\emph{\small{Runable Example}}}
\begin{lstlisting}
  package csjava;

  class tA1 implements Runnable {
    public void run{
      for(i=1;i=<5;i++)
        System.out.println{"i = " +i};
    }
  }

  class tA2 implements Runnable {
    public void run{
      for(j=1;j=<5;j++)
        System.out.println{"i = " +i};
    }
  }
  class tA3 implements Runnable {
    public void run{
      for(k=1;k=<5;k++)
        System.out.println{"i = " +i};
    }
  }

  public class threadExample {
    public static void main(String[] args) {
      tA1 A = new tA1():
      tA2 B = new tA2():
      tA3 C = new tA3():

      Thread t1 = new Thread(A);
      Thread t2 = new Thread(B);
      Thread t3 = new Thread(C);

      t1.start();
      t2.start();
      t3.start();
    }
  }
\end{lstlisting}

\newpage
\section{Questions}
\subsection{Demonstrate Exception handling using predefined atleast 3 predefined exceptions}
\subsection{How to create user defined exceptions. Explain with examples}

\end{document}%}}}
