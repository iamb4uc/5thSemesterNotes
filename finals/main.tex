\documentclass[twocolumn, 12pt, a4paper]{article}

%{{{
% Packages{{{
\usepackage{mathpazo}
\usepackage{graphicx}
  \graphicspath{ {./images/} }
\usepackage[T1]{fontenc}
  \linespread{1.2}
\usepackage{microtype}
\usepackage[english]{babel}

\usepackage[margin=0.5in, columnsep=10mm]{geometry}

\usepackage{enumitem}
  \setlist[enumerate]{itemsep=0px, left=0pt}
  \setlist[itemize]{itemsep=0px, left=0pt}

\usepackage{titlesec}
\titleformat{\part}[block]{\Huge\slshape\bfseries\centering}{}{0em}{}
\titleformat{\section}[block]{\LARGE\scshape\bfseries\centering}{}{0em}{}
\titleformat{\subsection}[block]{\Large\bfseries}{}{0em}{}
\titleformat{\subsubsection}[block]{\bfseries}{}{0em}{}
\usepackage{titling}
\usepackage{tcolorbox}

\usepackage{fancyhdr} % Headers and footers
\pagestyle{fancy} % All pages have headers and footers
\fancyhead{} % Blank out the default header
\fancyfoot{} % Blank out the default footer
\renewcommand{\headrule}{}
%}}}

% Title{{{
\setlength{\droptitle}{-4\baselineskip}

\pretitle{\begin{center}\Huge\bfseries}
\posttitle{\end{center}}
\title{Management Information System}
\author{ by
  \textsc{Swapnil}%\\[1px]
}
\date{}
%}}}}}}

\begin{document}
\maketitle
\thispagestyle{empty}

\begin{tcolorbox}
\part{Unit 1 and 2}
\end{tcolorbox}

\section{Information System}
An information system is a set of computer-based tools for collecting, storing
and processing of data in our world.

Businesses and other organizations rely on Information System to:

\begin{enumerate}
  \item Carry out and manage other operations.
  \item Interact with their customers and suppliers.
  \item Compete in the market place.
\end{enumerate}

For example:
\begin{enumerate}
  \item Corporation use Information systems to:
    \begin{enumerate}
      \item reach-out to their potential customers through targeted
        advertisements, messages over the web.
      \item process financial accounts.
      \item manage their workforce and inventory.
    \end{enumerate}
  \item Government use Information systems to:
    \begin{enumerate}
      \item provide services to their citizens.
      \item manage the economy.
      \item to collect taxes.
    \end{enumerate}
\end{enumerate}

Digital goods such as electronic books and software and online services such as
e-businesses, e-commerce, and social media are all provided and operated by an
information system. A typical information system uses a database to store its
data and often these proves have an user interface, where we the users issue
commands and can see the results.

Information systems plays a huge rule in all the aspects of our life in our
constantly connected digital world.

\subsection{Types of information systems}
\subsubsection{Transaction Processing Systems (TPS)}
These systems are used to process and record business transactions, such as
sales and purchases. They are used to capture and store data, such as financial
transactions, customer information, and inventory data.

\subsubsection{Management Information Systems (MIS)}
These systems provide managers with the information they need to make
decisions. They are used to generate reports and analyze data, such as sales
figures, customer demographics, and financial performance.

\subsubsection{Decision support systems (DSS)}
These systems are used to help managers make decisions. They are used to
analyze data, such as sales figures, customer demographics, and financial
performance, to identify trends and patterns.

Some important characteristics of DSS are:
\begin{enumerate}
  \item Adaptability and flexibility.
  \item High level of interactivity.
  \item Ease of use.
  \item Efficiency and effectiveness.
  \item Complete control by decision-makers.
  \item Ease of development.
\end{enumerate}

\subsubsection{Expert System}
These systems are used to provide expert advice in a specific area. They are
used to analyze data and make decisions based on that data, such as diagnosing 
medical conditions or identifying potential fraud.

\subsection{How information systems help grow businesses?}
Organizations use information systems to run most of their operations from 
sales and management to manufacturing and inventory management to human
resource management and finance.

Information systems are engineers to handle all the data and interactions
necessary to keep a business going.

Many of the application and services made possible within the internet can be
offered to an organization's users establishing an intranet.

An intranet is a network within an organization that uses network technology
to collect, store and share useful information within a business. An extranet
connects intranets of business partners so communication between organisations
can be possible. Some of these systems such as an electronic fund transfer and
emails have been used in traditional businesses as well as e-commerce.

Overall, in current age of ever growing demand and competition in between
different businesses, information systems plays a huge role in defining new 
standards and good customer experiences. Without a well developed information
system, managing a business is quite difficult.


\section{Balanced MIS}
A balanced Management Information System (MIS) is a system that provides
managers with the right information, in the right format, at the right time,
to make informed decisions. It is a system that is designed to balance the
needs of different users, such as top management, middle management, and
operational staff, by providing them with the information they need to perform
their specific tasks.

A balanced MIS includes a combination of both internal and external
information sources, such as financial data, operational data, market data,
and customer data. It also includes a variety of different reporting and
analytical tools, such as dashboards, reports, and scorecards, to help
managers make sense of the data and identify trends and patterns.

One of the key features of a balanced MIS is that it is flexible and
adaptable. It is designed to be able to grow and change as the needs of the
business change. It is also designed to be easy to use, so that managers can
quickly and easily access the information they need.

Overall, a balanced MIS is an essential tool for managers to make informed
decisions, monitor performance, and achieve strategic goals. It provides a
comprehensive view of the business and helps managers to identify and respond 
to opportunities and challenges in a timely manner.

\section{MIS of school/college management}
A Management Information System (MIS) plays a critical role in the management 
of schools and colleges. It is a computerized system that helps administrators
and teachers to collect, store, and analyze data to make informed decisions.

Here are a few ways in which an MIS can benefit school/college management:
\begin{enumerate}
  \item \textbf{Student Information Management:} An MIS can help
    administrators to manage student information, such as enrollment,
    attendance, grades, and disciplinary records. This information can be
    easily accessed and updated, which can help to streamline administrative
    tasks.
  
  \item \textbf{Curriculum Management:} An MIS can help teachers to manage the
    curriculum, such as creating lesson plans and tracking student progress.
    This can help to ensure that students are receiving the education they
    need to succeed.
  
  \item \textbf{Financial Management:} An MIS can help to manage the financial
    aspects of a school or college, such as budgeting, accounting, and
    reporting. This can help to ensure that resources are being used
    effectively and efficiently.
  
  \item \textbf{Reporting and Analytics:} An MIS can provide teachers and 
    administrators with detailed reports and analytics on student performance,
    attendance, and other key metrics. This information can be used to
    identify areas of improvement and to make data-driven decisions.
  
  \item \textbf{Communication and Collaboration:} An MIS can facilitate
    communication and collaboration among teachers, administrators, and
    students. For example, teachers can use the system to communicate with
    parents, administrators can use it to communicate with teachers, and
    students can use it to communicate with teachers.
\end{enumerate}

Overall, an MIS is a powerful tool that can help schools and colleges to manage their operations more efficiently and effectively. It can help to improve student performance, streamline administrative tasks, and ensure that resources are being used effectively.

\subsection{Hardware and Software requirements}
\subsubsection{Hardware Interface}
Following are the basic requirements to run a Management Information System 
for a school or a college:
\begin{itemize}
  \item \textbf{A LAN connection} for interacting with the database and local 
    computers.
  \item TCP/IP protocol for communicating with local hosts.
  \item A system with a
    \begin{itemize}
      \item P4 processor
      \item 1 Gigabyte of RAM for database memory.
    \end{itemize}
\end{itemize}

\subsubsection{Software Interfaces}
\begin{itemize}
  \item An operating system.
  \item An IDE for writing programs.
  \item Programming languages like
    \begin{itemize}
      \item Microsoft .Net 3.5 and C\# .Net 3.5 for writing the code for the
        project
      \item ASP.Net 3.5 for creating the web pages
      \item Oracle SQl, mysql, and Microsoft access and other query languages
        for local and global databases
    \end{itemize}
    are used in schools and colleges along with various Graphical User
    Interface for login screens and interacting with the database.
  \item An IDE for writing programs.
\end{itemize}

\section{E-business}
Any kind of business or commercial transaction that includes sharing
information accross the internet. It constitutes the exchange of products \& 
services between businesses, groups, and individuals and can be seen as one of
the essential activities of any business.

\section{E-commerce}
E-commerce, short for electronic commerce, is the buying and selling of goods
and services over the internet. It refers to any commercial transaction that is
conducted online, whether it is the sale of physical products, digital
products, or services. E-commerce includes a wide range of activities such as
online shopping, online banking, online ticketing, and online reservations.

E-commerce has revolutionized the way people shop and conduct business. It has
made it possible for businesses to reach a global customer base, and for
consumers to shop from the comfort of their own homes. E-commerce has also made
it easier for businesses to track sales, customer behavior, and inventory,
which has allowed them to make better decisions and improve their overall
operations.

There are different types of e-commerce: B2B (business-to-business), B2C
(business-to-consumer), C2B (consumer-to-business) and C2C
(consumer-to-consumer).

E-commerce has grown rapidly in recent years due to the increasing popularity
of online shopping and the widespread use of mobile devices. This growth has
been driven by several factors such as the availability of high-speed internet,
the increasing use of mobile devices, and the growing trend of online shopping.

Overall, e-commerce has changed the way businesses operate and has made it
easier for consumers to shop, pay and access services online.

\subsection{Different Types of E-commerce}
There are four main types of e-commerce:

\begin{enumerate}
  \item \textbf{B2B (Business-to-Business):} This type of e-commerce involves
    transactions between businesses, such as a supplier selling products to a
    retailer.
  \item \textbf{B2C (Business-to-Consumer):} This type of e-commerce involves
    transactions between businesses and consumers, such as a retailer selling
    products to a customer. This is the most common type of e-commerce and
    includes online shopping sites, such as Amazon or Walmart.
  \item \textbf{C2B (Consumer-to-Business):} This type of e-commerce involves
    transactions where consumers sell products or services to businesses, such
    as a freelancer offering services to a company.
  \item \textbf{C2C (Consumer-to-Consumer):} This type of e-commerce involves
    transactions between consumers, such as buying and selling items on a
    marketplace like eBay or Etsy.
\end{enumerate}

It's easy to understand all four types, B2B is business to business, B2C is
business to customer, C2B is customer to business and C2C is customer to
customer. Each type of e-commerce is designed to serve a specific need or
purpose, and businesses can choose the type that best suits their needs.

\section{Site Licence}
A site license is a type of software license that allows a certain number of
users within a specific location or organization to use a particular software
program. Site licenses are usually sold to businesses, schools, and other
organizations that have multiple users who need to access the software.

A site license is typically based on a one-time payment, which gives the
organization or location the right to use the software on an unlimited number
of computers. This is in contrast to individual licenses, which are sold to
individual users and typically require a separate payment for each user.

With a site license, the organization or location can install the software on
all of its computers, and users can access the software without having to
purchase individual licenses. This can be a cost-effective option for
organizations with many users, as it eliminates the need to purchase multiple
individual licenses.

However, it is important to note that site licenses usually come with specific
terms and conditions, such as the number of users allowed and the specific
location or organization that the license applies to. It's also important to
note that Site licenses are usually non-transferable, meaning that the license
can't be used by different organization or location.

In summary, a site license is a type of software license that allows a certain 
number of users within a specific location or organization to use a particular 
software program, it's cost-effective and can be used by an unlimited number of
users.

\subsection{Network multi-license mean}
A network multi-license is a type of software license that allows multiple
users within a network or organization to access and use a particular software
program. It's based on a one-time payment, which gives the organization the
right to use the software on a certain number of computers within the network,
it's cost-effective and eliminates the need to purchase multiple individual
licenses. However, it usually comes with specific terms and conditions, such as
the number of users allowed and the specific network or organization that the
license applies to.

\subsection{What Does Public Domain Software Mean?}
Public domain software refers to software that is not protected by copyright
and is available for free use, modification, and distribution. This means that
anyone can use, copy, modify, and distribute the software without any legal or
financial constraints. Public domain software is often created and released by
individuals, organizations, or government agencies, who choose to relinquish
their rights to the software and make it freely available to the public.

Some examples of public domain software include:

\begin{enumerate}
  \item Linux: a popular open-source operating system that is freely available
    to use, modify, and distribute. 
  \item Apache: an open-source web server that is widely used to host websites
    and web applications. 
  \item Python: a popular open-source programming language that is widely used
    for a variety of applications. 
  \item Gnu: a collection of open-source software tools that are widely used
    for a variety of applications. 
\end{enumerate}

Public domain software can be an attractive option for businesses,
organizations, and individuals who want to use or modify software without
incurring any licensing costs. However, it's important to note that public
domain software is not always supported and may not have the same level of
security and reliability as commercial software.

\section{DBMS}

In mainframe and mid-range computer systems, a DBMS is considered an important system software package that controls the development, use and maintenance of the database of computer using organisations. A DBMS programs helps organisation use their integrated collection of data records and files known as databases.

A DBMS also simplifies the process of retrieving information from databases in the form of displays and reports. Instead of having to write computer products to extract information encloser can ask simple questions in a query language. Thus many DBMS packages provide fourth generation language and other application development feature.

\subsection{Feature, Advantages of DBMS.}
DBMS is a set of program that allows access, retrieval and use of that data by considering appropriate security measures. The DBMS is really useful for better data integration an its security.

Advantages of DBMS:
\begin{enumerate}
  
\item \textbf{Data Integrity}: A DBMS ensures data integrity by enforcing rules and constraints to prevent data inconsistencies and errors. 
\item \textbf{Data Security}: DBMS provides security features such as user authentication and access control to protect data from unauthorized access and manipulation. 
\item \textbf{Data Backup and Recovery}: DBMS provides the capability to backup and recover data in case of system failure or data loss. 
\item \textbf{Data Consistency}: DBMS ensures data consistency by ensuring that data is accurate and up-to-date across different applications and users. 
\item \textbf{Data sharing}: DBMS allows multiple users to access and manipulate data simultaneously, which improves collaboration and data sharing. 
\item \textbf{Data Scalability}: DBMS can handle large amounts of data and can be easily scaled to accommodate growth and changing needs. 
\item \textbf{Improved Data Access}: DBMS provides powerful query and data retrieval capabilities, which allows users to easily access and analyze data. 
\item \textbf{Data independence}: DBMS allows users to access data in a consistent way regardless of the underlying physical structure of the data. 
\item \textbf{Reduced data redundancy}: DBMS helps to reduce data redundancy by storing data only once and providing access to it as needed. 
\item \textbf{Data validation}: DBMS checks the data before storing it in the database, which ensures that the data is accurate and complete.

\end{enumerate}

\subsection{Application of DBMS}
\subsubsection{Customer Relationship Management}
CRM database can help small business manage its customers. A CRM database organizes all the information a company has about its accounts, contacts, leads and opportunities.

\subsubsection{Inventory Tracking Database}
It helps a retail business manage how much inventory is in a warehouse, in a storage room and on store shelves.

\subsubsection{Payroll and Scheduling Database}
It simplifies scheduling and help prevent payroll errors.
An employee database contains such fields as hourly wage, salary or commission, tax withholding rates, year-to-date income and accrued vacation time.

\subsubsection{Business Data Analysis}
Databases make the process of analyzing data and predicting future trends easier.




\onecolumn

\section{Difference between TPS and MIS}
\begin{table}[ht]
  \resizebox{\textwidth}{!}{
  \begin{tabular}{p{2cm}p{8cm}p{7cm}}
  &
  \textbf{TPS}&
  \textbf{MIS}\\
 
  \emph{Input}&
  Transaction/events &
  Output from TPS\\

  \emph{Output}&
    Data entry, listing, sorting, merging and updating. &
    Routine reports, simple models, low level analysis.\\

  \emph{Users}&
    Operational personals, lower-level managers, supervisors. &
    Middle-level manager\\

  \emph{Goal}&
    Records and processes transactions. &
    Production of summary and exception reports.\\

  \emph{Decision \& support} &
    Provides decision support to lower-level managers &
    Provide decision supports to tactical-level managers\\

  \end{tabular}}
\end{table}

\end{document}
